\documentclass[10pt]{article}

\usepackage{macros}

%Formatting
\linespread{1.25}

\usepackage{parskip}
\setlength{\parindent}{18pt}
\setlength{\parindent}{0cm}

\setcounter{tocdepth}{2}
\setcounter{secnumdepth}{5}

\title{}

\author{Brian R. Williams}

\def\brian{\textcolor{blue}{BW: }\textcolor{blue}}

\begin{document}
\maketitle

There is a deep connection between gauge theory and quantum groups.
Perhaps the most ubiquitous is the relationship between the category of line operators in Chern-Simons theory on three manifolds and the braided monoidal category of representations of the quantum group.
In the seminal work, Costello \cite{CosInt, CosYangian} has shown how certain infinite dimensional quantum groups, Yangians, arise from the algebra of operators of a {\em four} dimensional (supersymmetric) gauge theory.
The key element 

Factorization algebras, Koszul duality, 


\begin{itemize}
\item An introduction to factorization algebras valued in symmetric monoidal $\infty$-categories. Lurie's result that locally constant factorization algebras recover algebras over the operad of little disks.
Modules for $E_n$-algebras.\cite{CG1, LurieHA, FrancisAyalaTopMan}
\item Holomorphic factorization algebras and their relationship to chiral and vertex algebras. \cite{CG1, BD}
\item Koszul duality for $E_n$-algebras.
Tamarkin's \cite{Tamarkin} result that the Koszul dual of an $E_2$ algebra is a Hopf algebra.
The Hochschild homology of $E_n$ algebras. \cite{FrancisAyala, ...}
\item An introduction to quantum groups and the Yangian.
Drinfeld's universal $R$-matrix. \cite{Etingof, ChariPressley}
Relationship to integrable systems and lattice models.
\item The classical moduli space of holomorphic partially flat connections on complex surfaces.\footnote{Roughly, a holomorphic partially flat connecton on a complex manifold of the form $X \times Y$ is a holomorphic connection that is holomorphically flat in the $Y$-direction.}
The Batalin-Vilkovisky formalism and quantization.
Costello-Gwilliam's result that the operators of a gauge theory (or any QFT) form a factorization algebra. \cite{CosBook, CG2, CosYangian}
\item 
\end{itemize}

%\tableofcontents

%\section*{Acknowledgements} 

%\input{first section}
%\bibliographystyle{alpha}
%\bibliographystyle{spmpsci}  
%\bibliography{thesis}

\end{document}