\documentclass[10pt]{amsart}

\usepackage{macros}

%Formatting
\linespread{1.25}

\usepackage{parskip}
\setlength{\parindent}{18pt}
\setlength{\parindent}{0cm}

\setcounter{tocdepth}{2}
\setcounter{secnumdepth}{5}

\title{The Yangian and four-dimensional gauge theory}

\author{Brian R. Williams}
\date{} 

\def\brian{\textcolor{blue}{BW: }\textcolor{blue}}

\begin{document}
\maketitle

In very broad strokes, our goal is extract quantum groups from factorization algebras defined on certain manifolds.
For some examples the manifold is simply a copy of Euclidean space. 
The Yangian quantum group will come from a factorization algebra on $\CC \times \RR^2 = \RR^4$, where, in some sense, we remember the complex structure on the copy of Euclidean space $\CC = \RR^2$.

\section{What is a factorization algebra?}

Let $X$ be a manifold. (Think: spacetime.)

{A {\em prefactorization algebra} $\sF$ on $X$~is:}
{\small 
\begin{itemize}
\item {a vector space $\sF(U)$ for each open set $U \subset M$}

\item {a linear map $\sF(U) \rightarrow \sF(V)$ for each inclusion $U \subset V$}

\item {a linear map $\sF(U_1) \otimes \cdots \otimes \sF(U_n) \rightarrow \sF(V)$ for each $U_1,\ldots,U_n \subset V$ with the $U_i$ pairwise disjoint.}

\end{itemize}
}
satisfying
\begin{itemize}
\item {equivariance under relabeling} 
\item {associativity under composition: if $U_{i,1}\sqcup\cdots\sqcup U_{i,n_i}\subseteq V_i$ and $V_1\sqcup\cdots\sqcup V_k\subseteq W$, the following diagram commutes:}
\[
\xymatrix{
{\bigotimes}^{k}_{i=1}{\bigotimes}^{n_i}_{j=1}\sF(U_j) \ar[dr] \ar[rr] &   &{\bigotimes}^k_{i=1}\sF(V_i) \ar[dl]\\
&\sF(W)  &
}
\]
\end{itemize}

A prefactorization algebra $\sF$ on $X$ is a {\em factorization algebra} if it satisfies a certain gluing condition.
Note that this is {\bf not} the ordinary gluing condition saying that the underlying precosheaf $\sF$ determines is a cosheaf. 

The main result of Costello \cite{CosYangian} that we study in this seminar relates the geometry of a four-dimensional gauge theory to the algebra of quantum group deformations. 
The connection to gauge theory and factorization algebras is the guiding principal of Costello-Gwilliam \cite{CG1, CG2}: 
\begin{center}
{\em The observables, or operators, of a quantum field theory have the structure of a factorization algebra.}
\end{center}
On the geometric side, this says that once one takes into account a sufficient amount of locality on the manifold, the functions on the moduli space of connections (or more interestingly, a deformation thereof) form a factorization algebra in a natural way.  

The following picture portrays the general landscape of the theory of factorization algebras. 
\ben
\begin{tikzcd}
\left\{ \begin{array}{c} {\rm Chiral/vertex\; algebras} \\ {\rm (BD, BZF, \;etc..)} \end{array} \right\} \ar[dr, hook] & & \left\{ \begin{array}{c} {\rm Algebras\;over\;other} \\ {\rm (colored)\;operads.} \\ {\rm (Riem, cplx, conformal,...)} \end{array} \right\} \ar[dl, hook'] \\
& \left\{ \begin{array}{c} {\rm Factorization\;algebras} \end{array} \right\} & \\
& \left\{ \begin{array}{c} {\rm Locally\;constant} \\ {\rm factorization\;algebras} \end{array} \right\} \ar[u, hook] & \\
& \left\{ \begin{array}{c} \sE_n {\rm \;algebras} \end{array} \right\} \ar[u,equal, "{\rm Lurie}"] & .
\end{tikzcd}
\een
Perhaps the most well-known class of factorization algebras are the {\em locally constant} ones.
These are factorization algebras defined on a smooth manifold that roughly only depend on homeomorphism type of the manifold. 
In particular, this class of factorization algebras see no difference between open balls of varying radius. 
It is a theorem of Lurie \cite{LurieHA} that this class of factorization algebras is equivalent to the theory of $E_n$-algebras; algebras over the operad of little $n$-disks. 

This is quite a restrictive class of factorization algebras. 
In addition to locally constant factorization type, we will also encounter factorization algebras that {\em holomorphic}. 
These are algebras that are sensitive to complex structures. 

%Let's move back to the gauge theory at hand. 
%From a geometric or physical point of view, one views the fields of the theory as coming from the smooth sections of some vector bundle. 
%For the Yangian theory, the fields take the following form
%\ben
%\Omega^{0,*}(\Sigma) \tensor \Omega^*(S) \tensor \fg[1] .
%\een 
%All the interesting moduli space data is encoded from the dg Lie algebra structure on the object above, together with some shifted symplectic pairing. 

\begin{eg}
If $E$ is any (graded) vector bundle on a manifold $X$, we can form the infinite symmetric product
\ben
\Sym^*(E) = \ul{\CC} \oplus E \oplus \Sym^2(E) \oplus \cdots .
\een 
This is an infinite dimensional vector bundle on $X$, with a nice topology coming from the natural filtration by symmetric degree.
\end{eg}

\begin{fact} For any vector bundle, the cosheaf of compactly supported sections of $\Sym^*(E)$:
\ben
U \subset X \mapsto \Gamma_c(U, \Sym^*(E))
\een
has the structure of a factorization algebra.
\end{fact}

All of the examples of factorization algebras we encounter is this seminar arise as deformations of ones of this form (or by replacing compactly supported sections by distributional compactly supported sections).

\subsection{The observables of a QFT}

It is completely geometric that the {\em classical} observables (function on sections of some vector bundle) have the structure of a factorization algebra. 
Generically, these factorization algebras have the form
\ben
\Obs^{cl} = \left(\Sym( \sE^\vee), Q^{cl}\right),
\een
where $Q^{cl}$ is the classical differential. 
In the contexts we have addressed, the classical differential behaves like a Chevalley-Eilenberg differential computing Lie algebra cohomology. 

The Batalin-Vilkovisky formalism is an approach developed by Costello in \cite{CosBook} to construct the path integral in a rigorous way.
At the level of observables, a quantization produces a factorization algebra of the form
\ben
\Obs^q = \left(\Sym(\sE^\vee) [[\hbar]] , Q_\hbar = Q^{cl} + \hbar Q^{(1)} + \hbar^2 Q^{(2)} + \cdots\right),
\een
where $Q_\hbar$ is the {\em quantum} differential.

There is a precise analogy of the BV formalism with ordinary deformation quantization.
The classical observables are always equipped with some structure of a (shifted) Poisson algebra. 
The factorization algebra of quantum observables deforms this Poisson structure. 

Extreme care must be taken when defining the quantum differential rigorously.
This is related to the difficulties in defining the path integral.  
For instance, naively the terms in the differential $Q^{(i)}$ for $i \geq 1$ have support {\em everywhere} on the spacetime manifold. 
This makes the process of defining the factorization algebra, which is really a local object, a tricky one. 

%\subsection{Koszul duality}
%
%\def\Mod{\mathsf{Mod}}
%\def\RRHom{\RR{\rm Hom}}
%\def\Ext{{\rm Ext}}
%
%Koszul duality is an algebraic phenomena that occurs in the context of augmentated algebras.
%Let $A$ be an associative dg algebra. 
%An augmentation of a $\CC$-algebra $A$ is a homomorphism of algebras $\epsilon : A \to \CC$. 
%This augmentation equips $\CC$ with the structure of an $A$-module that we denote $\CC_{\epsilon}$. 
%
%Categorically, Koszul duality asserts an equivalence between certain categories of modules.
%Since $A$ is a dg algebra, its category of dg modules $\Mod_A$ has the structure of a dg category. 
%This means, given dg modules $M,N$ there is a cochain complex of homomorphisms from $M$ to $N$ that we denote by 
%\ben
%\RRHom_A (M, N) .
%\een
%%The cohomology of this complex is denoted $\Ext_A(M,N)$. 
%
%Since an augmentation makes $\CC = \CC_\epsilon$ into an $A$-module, we can consider the cochain complex
%\ben
%A^{!} = \RRHom_A(\CC_{\epsilon}, \CC_{\epsilon})
%\een
%Since the (derived) endomorphisms of any module have the structure of a dg algebra, the cochain complex $A^{!}$ also has the structure of a dg algebra. 
%This dg algebra is called the {\em Koszul dual} of $A$. 
%
%\begin{eg} Let $(V, \d)$ be any cochain complex and consider the {\em commutative} dg algebra $\Sym^*(V) = \oplus_{n \geq 0} \Sym^n(V)$.
%Then, $\Sym^*(V)$ has a natural augmentation $\Sym^*(V) \to \Sym^0(V) = \CC$.  
%The Koszul dual of $\Sym^*(V)$ with respect to this augmentation is equal to the exterior algebra
%\ben
%\Sym^*(V^\vee[-1]) 
%\een
%where $V^\vee [-1]$ is the dual vector space placed in cohomological degree $+1$. 
%Note that if we forget the grading, this dg algebra is the usual exterior algebra $\wedge^* V^\vee$. 
%
%As a deformation of this example, one can show that for any dg Lie algebra $\fg$, the Koszul dual of the universal enveloping algebra $U(\fg)$ is the dg commutative algebra of Chevalley-Eilenberg cochains $\clie^*(\fg)$. 
%\end{eg}
%
%The formal statement of Koszul duality asserts an equivalence of categories between some subcategory of dg $A$-modules and some subcategory of dg $A^{!}$-modules.
%
%What is the relationship between this algebraic manipulation of Koszul duality and gauge theory? 
%Consider a theory defined on a manifold of the form
%\ben
%\RR_t \times \RR^n .
%\een
%Assume, further, that the theory is topological in the $\RR_t$ direction. 
%Equivalently, the factorization algebra restricted to $\RR_t \times \{x\}$ is locally constant for all $x \in \RR^n$.
%Let $\sA$ be the factorization algebra on $\RR_t$ restricted to $\RR_t \times \{0\}$. 
%Then, by local constancy, we know that $\sA$ is equivalent to an $\sE_1$, or dg associative, algebra. 
%Suppose, in addition 
%
%We study the problem of coupling a ``line operator" placed at $\RR_t \times \{0\}$ to the theory on $\RR_t \times \RR^n$. 
%Factorization along the line gives such an operator the structure of an associative algebra $\sB$. 
%It turns out, that such a coupling is equivalent to a Maurer-Cartan element in the dg algebra 
%\ben
%\sA \tensor \sB .
%\een 
%
%\begin{prop}[Lurie] Suppose $\sA$ is augmented. 
%Then, Maurer-Cartan elements $\alpha$ in the dg algebra $\sA \tensor \sB$ such that $\epsilon(\alpha) = 0$ are equivalent to maps of algebras
%\ben
%\alpha : \sA^! \to \sB .
%\een
%\end{prop}


\section{Prelude: Chern-Simons and quantum groups}

Chern-Simons theory is a three-dimensional topological field theory consisting of flat connections of a principal $G$-bundle. 
The moduli space of the solutions to the classical equations of motion of Chern-Simons on a three-manifold $M$ is ${\rm Flat}_G(M)$, the moduli space of $G$-local systems on $M$. 
If $A$ represents a connection on a principal $G$-bundle, flatness is the equation
\ben 
F(A) = \d A + \frac{1}{2} [A,A] = 0 .
\een 
Witten \cite{WittenJones} showed that quantum Chern-Simons theory is intimately connected to knot invariants, specifically, the Jones polynomial. 
As a result, Chern-Simons theory is indirectly related to the theory of quantum groups. 

In the style of physics, Chern-Simons theory is described by an action functional depending on some connections of principal $G$-bundles over a three-manifold. 
The critical values of the action functional precisely correspond to flat $G$-bundles. 
Two flat $G$-bundles are said to be equivalent if they differ by a $G$-valued gauge transformation. 
The moduli space of flat $G$-bundles is the space of flat connections modulo gauge. 
Of course, this is a highly singular object, so one must consider it as an appropriate stacky object. 

Suppose the three-manifold we are studying is of the form $M = \Sigma \times \RR$, where $\Sigma$ is a Riemann surface. 
The moduli space of flat $G$-bundles that are {\em constant} in the direction of $\RR$ is of the form
\begin{align*}
{\rm Flat}_G(\Sigma) & = \{{\rm Flat \; }G{\rm -bundles\;on\;}\Sigma\} \\ & = {\rm Hom}(\pi_1(\Sigma), G) / G
\end{align*}
We will assume that $G$ is a complex semi-simple group.

\begin{fact}
The space ${\rm Flat}_G(\Sigma)$ is a {\em symplectic} manifold. 
\end{fact} 

If $A$ is some reference flat connection for a principal $G$-bundle $P$ on $\Sigma$, then the tangent space of the moduli space of flat $G$-bundles at $A$ is the first order deformations of $A$ in the space of all flat $G$-bundles. 
We must still take the quotient by gauge transformations. 
That is:
\ben
T_A {\rm Flat}_G(\Sigma) = \{\Tilde{A} \in \Omega^1(\Sigma , \fg_P) \; | \; A + \epsilon \Tilde{A} {\rm \; is\;flat}\} / {\rm gauge}
\een
Here, $\epsilon$ is a formal parameter such that $\epsilon^2 = 0$. 
Also, $\fg_A$ is the adjoint bundle of $P$, it is a bundle of Lie algebras with fiber $\fg = {\rm Lie}(G)$. 
The notation $\fg_A$ is to remind us that we are equipping it with the flat connection $A$. 

Since $A$ is flat, the condition that $A + \epsilon \Tilde{A}$ is flat is equivalent to 
\ben
\d_A \Tilde{A} = \d \Tilde{A} + [\Tilde{A}, A] = 0 .
\een 
Here, $\d_A = \d + [A,-]$ is the covariant derivative with respect to $A$. 
Moreover, the gauge transformations $f \in \Gamma(\Sigma, \fg_P)$ act by
\ben
\Tilde{A} \mapsto \d_A f
\een
Thus, we see that $T_A{\rm Flat}_G(\Sigma)$ is precisely the cohomology group $H^1(\Sigma, \fg_P)$. 

If $\Tilde{A}, \Tilde{A}' \in T_A H^1(\Sigma, \fg_P)$ we define the symplectic form
\ben
\omega_A(\Tilde{A}, \Tilde{A}') = \int_\Sigma \<\Tilde{A}, \Tilde{A}'\>_\fg
\een
where $\<-,-\>_\fg$ is the Killing form on $\fg$. 
Note that the integrand on the RHS is a two-form so it can be integrated.

The full, derived tangent space, can be identified with $\Omega^*(\Sigma, \fg_P)[1]$ equipped with the differential $\d_A = \d + [A,-]$. 
Note that $\Omega^*(\Sigma, \fg_P)$ has the structure of a dg Lie algebra, and the Maurer-Cartan elements are precisely deformations of the flat connection $A$. 
This is a generic property of formal moduli, and more relevant to us, perturbative field theories. 

Quantum Chern-Simons on $\Sigma \times \RR$ essentially reduces to studying deformation quantization for the symplectic manifold ${\rm Flat}_G(\Sigma)$. 
The operators of the classical theory are just the functions on the symplectic manifold $\sO({\rm Flat}_G(\Sigma))$.
Symplectic structure equips this with a Poisson bracket. 
A deformation quantization is an algebra of the form $\sO_{\hbar}({\rm Flat}_G(\Sigma))$ whose associative product satisfies
\ben
a, b \in \sO_\hbar \implies \lim_{\hbar \to 0} \frac{1}{\hbar} [a,b] = \{a,b\} .
\een

If we work near the trivial connection on $\RR^3 = \RR^2 \times \RR$ the moduli space ${\rm Flat}_G(\RR^2)$ is described by the dg Lie algebra $\Omega^*(\RR^2, \fg)$. 
Functions on the moduli space are modeled on the Chevalley-Eilenberg cochain complex
\ben
\sO({\rm Flat}_G(\RR^2)) = C^*(\Omega^*(\RR^2, \fg))
\een 
Note that as a cochain complex, there is a quasi-isomorphism $C^*(\Omega^*(\RR^2, \fg)) \simeq C^*(\fg)$. 

\ben
\begin{tikzcd}
\sO_{\hbar} ({\rm Flat}_G(\Sigma)) \ar[d, "\hbar \to 0"'] & \simeq & C^*_{\hbar}(\fg) \arrow[d, "\hbar \to 0"'] \arrow[r, leftrightarrow, "{\rm Koszul}"] & U_\hbar(\fg)  \arrow[d,"\hbar \to 0"]  \\
\sO ({\rm Flat}_G(\Sigma)) & \simeq & C^*(\fg) \arrow[r, leftrightarrow, "{\rm Koszul}"'] & U(\fg) .
\end{tikzcd}
\een

This picture hints at a more direct relationship between Chern-Simons theory and the quantum group, without passing through the theory of knot invariants. 
Even so, knot invariants have a natural physical interpretation as line operators in the three-dimensional theory. 
This picture is compatible with the approach we take to obtain the quantum group via factorization methods.

\subsection{Example: Chern-Simons factorization algebra}

We've been a bit naive here, and never discussed the actual algebraic structure the deformation $C^*_{\hbar}(\fg)$ possesses.
To really understand this, we need to appeal to the theory of factorization algebras. 
 
On a $3$-manifold, the fields of Chern-Simons theory are also of this form:
\ben
\Omega^*(M^3) \tensor \fg [1] .
\een 
By definition, the classical observables supported on an open set $U$ are defined as the functions on fields supported on $U$. 
Thus, for Chern-Simons, a linear observable on $U$ looks like a linear map of the form
\ben
O : \Omega^{*}(M) \tensor \fg [1] \to \CC .
\een 
Using Poincar\'{e} duality, we can read off that the space of all observables supported on $U$ is
\ben
\Obs_{CS}^{cl} (U) = \Sym^* \left( (\Omega^*(U) \tensor \fg[1])^\vee\right) = \Sym^* \left(\Omega_c^*(U) \tensor \fg^\vee [2] \right) .
\een
This cochain complex is equipped with the Chevalley-Eilenberg differential coming from the dg Lie algebra $\Omega^*(M) \tensor \fg$.

\begin{rmk} Strictly speaking, we should really use the {\em distributional} compactly supported sections.
\end{rmk}

Chern-Simons is the typical example of a topological field theory. 
In this setting, this means that the factorization algebra 
\ben
U \mapsto \Obs^{cl}(U)
\een
is locally constant.
Indeed, for each disk $D$ the classical observables are quasi-isomorphic to the Chevalley-Eilenberg complex $\Obs^{cl}(D) \simeq C^*(\fg)$. 
On the three-manifold $M = \RR^3$ this implies, by Lurie's theorem, that the observables determine an $\sE_3$-algebra.
In fact, classically, this is actually a commutative (or $\sE_\infty$-algebra). 

BV quantization picks out a deformation $\Obs^q$ of $\Obs^{cl}$ as locally constant factorization algebras on $\RR^3$. 
Equivalently, BV quantization determines a deformation $C^*_\hbar(\fg)$ of $C^*(\fg)$ {\em as an $\sE_3$-algebra}. 

On one hand, we are claiming that the Koszul dual of this deformation of the quantum group.
For Chern-Simons, the quantum group can be shown to also arise in the following, more geometric, way. 
This is currently work in progress of Costello-Francis-Gwilliam. 
The $E_3$-algebra has an associated category of modules $\Mod_{\Obs_{CS}}$. 
By the general technology of factorization algebras, this category is equipped with an $\sE_2$-monoidal structure (i.e. an $\sE_2$-algebra in the category of categories). 
Concretely, this means there is a two-dimensional space in which one can take the tensor product of two modules. 
In geometric terms, we can view a module for the $\sE_3$-algebra as living on a line in $\RR^3$.
The tensor product corresponds to an ``operator product" of lines.
Since the space orthogonal to a line in $\RR^3$ is a copy of $\RR^2$, this gives two directions in which a tensor product can take place. 
In more classical terms, $\sE_2$-monoidal categories arise as {\em braided monoidal} categories. 
It is known that the modules for a quasi-triangular Hopf algebra form a braided monoidal category. 
The connection to Chern-Simons is the following:

\begin{thm}[Costello-Francis-Gwilliam]
The braided monoidal category of finite dimensional representations of the quantum group $U_{\hbar} \fg$ is equivalent to a subcategory of the braided monoidal category of modules for the $\sE_3$-algebra $\Obs_{CS}$. 
\end{thm}

\section{The four-dimensional theory}

The four-dimensional gauge theory we study is formally very similar to three-dimensional Chern-Simons theory. 

We start with a four-manifold of the form $\Sigma \times S$ where $\Sigma$ is a Riemann surface and $S$ is a real two-dimensional manifold. 
Throughout the course of this seminar we will take $\Sigma$ to be either $\CC$ or an elliptic curve $E$. 

We also fix the data of a {\em holomorphic} one-form $\omega \in \Omega^{1,hol}_{cl}(\Sigma)$ that is no-where vanishing. 
The moduli space of the four-dimensional gauge theory consists of connections $A$ satisfying the following modified version of flatness
\ben
\omega \wedge F(A) = \omega \wedge \left( \d A + \frac{1}{2}[A,A]\right) = 0 .
\een
Introduce local coordinates $(z,\zbar)$ on $\Sigma$ and $(w, \wbar)$ on $S$. 
Locally, a connection one-form looks like
\ben
A = A_{z} \d z + A_{\zbar} \d \zbar + A_{w} \d w + A_{\wbar} \d \wbar
\een
In the flatness equation, note that the component $A_z$ never appears. 
Thus, we may as well restrict ourselves to connections of the form
\be\label{connection}
A = A_{\zbar} \d \zbar + A_{w} \d w + A_{\wbar} \d \wbar .
\ee
We can then translate the flatness equation into the series of equations
\begin{align*}
\frac{\partial}{\partial w} A_{\wbar} + \frac{\partial}{\partial \wbar} A_w + [A_w, A_{\wbar}] & = 0 , \\
\frac{\partial}{\partial \wbar} A_{\zbar} + [A_{\wbar}, A_{\zbar}] & = 0 , \\
\frac{\partial}{\partial w} A_{\zbar} + [A_{w}, A_{\zbar}] & = 0 .
\end{align*}
We draw the following conclusions:
\begin{itemize}
\item
For each $w_0 \in S$ the connection one-form $A_{\zbar} \d \zbar$ restricted to $\Sigma \times \{w_0\}$ determines the structure of a holomorphic principal $G$-bundle on $\Sigma$ (since every $\fg$-valued $(0,1)$-form on a Riemann surface determines a holomorphic principal bundle). 
\item
The first equation above comes from the $\omega \Omega^{0}(\Sigma) \tensor \Omega^2(S)$ part of the curvature equation.
It implies that for each $z_0 \in \Sigma$ the connection one-form $A_w \d w + A_{\wbar} \d \wbar$ restricted to $\{z_0\} \times S$ is flat.
\item
The next two equations come from the component of the flatness equation living in $\omega \Omega^{0,1}(S) \tensor \Omega^1(\Sigma)$. 
Suppose that $\gamma : [0,1] \to S$ is a path from $\gamma(0) = w_0$ to $\gamma(1) = w_1$ in $S$. 
Then, we can view $\gamma$ as a map $\gamma : \Sigma \times [0,1] \to \Sigma \times S$. 
Then, these equation say that $\gamma^* A$ determines a homotopy between the holomorphic connections $(A_{\zbar} \d \zbar)|_{w_0}$ and $(A_{\zbar} \d \zbar)|_{w_1}$. 
In other words, the path $\gamma$ produces and isomorphism of holomorphic bundles over $\Sigma \times \{w_0\}$ and $\Sigma \times \{w_1\}$. 
Thus, we have a flat family of holomorphic bundles on $\Sigma$ parametrized by $S$. \footnote{
Dually, we can view the equations as saying we have a holomorphic family of flat bundles. }
\end{itemize}

The full space of fields of the four-dimensional theory is of the form
\ben
\Omega^{0,*}(\Sigma) \tensor \Omega^*(S) \tensor \fg[1] .
\een 
The shift down by one amounts to the fact that we want to view the fields in degree zero as connection one-forms. 
Check that the degree zero piece precisely consists of those connections (\ref{connection}). 

Most of this course will be concerned with the local setting where our four-manifold is of the form $\Sigma \times S = \CC \times \RR^2$. 
When $\Sigma = \CC$ and $S = \RR^2$ we will take for our holomorphic one-form $\omega = \d z$, unless otherwise noted. 

\begin{rmk} There are various variants of the theory just described.
There is a version of this gauge theory defined on any complex surface $X$ equipped with some extra structure. 
For instance, the theory makes sense on all holomorphic symplectic manifolds. 
\end{rmk}

\subsection{The operators of the four-dimensional gauge theory}

Analogously to the case of Chern-Simons, the main statement about quantum groups we aim to prove entails the operators of the four-dimensional gauge theory. 
Let's consider the gauge theory placed on the four-manifold of the form
\ben
\Sigma_z \times (S^1 \times \RR)_w .
\een 
Of course, there are no global coordinates $z,w$, we just continue to use the notation to indicate that the theory is holomorphic on the Riemann surface $\Sigma$, and topological on $S^1 \times \RR$. 
Just as in the case of Chern-Simons, we can look at solutions to the equations of motion that are constant along $\RR$, to get some moduli space of connections on $\Sigma \times S^1$. 

The analysis above says that for each point $t \in S^1$ the equation of motion dictate that we have a holomorphic bundle $V_t$ on $\Sigma$. 
As we go around the circle, monodromy along the flat connection in the $S^1$ direction implies that we obtain a non-trivial isomorphism of holomorphic bundles $\phi : V_0 \cong V_{2 \pi}$. 
Thus, the moduli space of solutions to the equations of motion on $\Sigma \times S^1$ is (roughly)
\ben
\sM = \{{\rm holomorphic \; bundles \; on} \; \Sigma \; {\rm with\;isomorphism}\} .
\een
This moduli can also be identified with the space of ``multiplicative Higgs bundles". 
It's also equivalent to the moduli of periodic monopoles on $\Sigma \times S^1$. 
This is a symplectic moduli space (in fact holomorphic symplectic), just as in the Chern-Simons case.

We will work formally near a trivial connection so that we may as well work on $\CC_z \times \RR_t$. 
Deformations of the symplectic moduli space near the trivial connection are controlled by the dg Lie algebra 
\ben
\Omega^{0,*}(\CC) \tensor \Omega^*(\RR) \tensor \fg .
\een 
Thus, if we perturb around the trivial connection, functions on the moduli space can be identified with
\ben
C^*(\Omega^{0,*} (\CC) \tensor \Omega^*(\RR) \tensor \fg) .
\een 
The symplectic structure equips this space with a Poisson bracket. 
The quantum gauge theory concerns the deformation quantization for the bracket. 
We have the following picture, which we hope bears resemblance to the three-dimensional Chern-Simons picture.
\ben
\begin{tikzcd}
\sO_{\hbar} (\sM) \ar[d, "\hbar \to 0"'] & \simeq & C^*(\Omega^{0,*} (\CC) \tensor \Omega^*(\RR) \tensor \fg) & C_{\hbar}^*(\fg [z]) \ar[l, hook',"\simeq"'] \arrow[d, "\hbar \to 0"'] \arrow[r, leftrightarrow, "{\rm Koszul}"] & \boxed{Y_{\hbar} (\fg)} \arrow[d,"\hbar \to 0"]  \\
\sO (\sM) & \simeq & C^*(\Omega^{0,*} (\CC) \tensor \Omega^*(\RR) \tensor \fg) & C^*(\fg[z]) \ar[l, hook', "\simeq"'] \arrow[r, leftrightarrow, "{\rm Koszul}"'] & U(\fg[[z]]) .
\end{tikzcd}
\een

Again, there is the problem of actually identifying the algebraic structure present in the deformation $C_\hbar^*(\fg[z])$. 
This is the main objective of the course.

\section{The plan for the course}

We will mostly consider the four-dimensional gauge theory on
\ben
\CC_z \times \RR^2_w .
\een
As pointed out in the last section, the quantum gauge theory produces a factorization algebra of observables $\Obs^q$ on $\CC \times \RR^2$. 
This is the main object of study. 

For each fixed $z_0$, one can restrict the factorization algebra $\Obs^q$ to the submanifold 
\ben
\{z_0\} \times \RR^2_w .
\een 
Denote this restriction by $\Obs_{z_0}^q$. 

\begin{lem}
The factorization algebra $\Obs^q_{z_0}$ is locally constant, and hence equivalent to an $\sE_2$-algebra.
\end{lem}

This is not just any $\sE_2$-algebra, it is actually an {\em augmented} $\sE_2$-algebra.
This will be a very important property.

We now do something kind of funny: any $\sE_2$-algebra can be considered as an $\sE_1$-algebra. 
Viewing the $\sE_2$-algebra as an algebra over the operad of little $2$-disks, we can simply project the $2$-disk onto a horizontal line.
Thus, there is a forgetful functor from $\sE_2$-algebras to $\sE_1$-algebras, which we can think of as dg associative algebras (up to homotopy).

\begin{thm}[\cite{Tamarkin}]
Suppose $A$ is an $\sE_2$-algebra that is augmented as an $\sE_1$-algebra. 
Then, its ($\sE_1$) Koszul dual algebra $A^{!}$ has the structure of a {\em Hopf algebra} (up to homotopy). 
\end{thm}

This theorem does not directly apply to our situation. 
Our algebras are built from infinite dimensional vector spaces, and to discuss Koszul duality rigorously we must really take 
into account some natural filtration. 
Nevertheless, there is a modification of Tamarkin's theorem that does apply. 

We can now state Costello's first theorem.

\begin{thm}[\cite{CosYangian}] 
For any $z_0 \in \CC$, the Koszul dual of the $\sE_2$-algebra $\Obs^q_{z_0}$ is equivalent to the Yangian $Y_\hbar(\fg)$. 
\end{thm}

\def\Fin{\mathsf{Fin}}
\def\Perf{\mathsf{Perf}}
This implies the following statement for categories of modules. 
If $\Fin(Y_\hbar\fg)$ denotes the monoidal dg category of finite-rank $Y_\hbar \fg$-modules, and $\Perf(\Obs_{z_0})$ denotes the monoidal dg category of perfect $\Obs_{z_0}$-modules, then by the general yoga of Koszul duality, there is an equivalence of monoiodal dg categories 
\ben
\Fin(Y_\hbar \fg) \simeq \Perf(\Obs_{z_0}) .
\een

Some formal manipulations in factorization algebras allow one to extract interesting consequences of this theorem. 
Here is one of them.
Consider placing the factorization algebra $\Obs_{z_0}$ on the punctured plane $(\RR^2_w)^\times \subset \RR^2_w$. 
Then, the pushforward of this along the radial projection
\ben
r : (\RR^2_w)^\times \to \RR_{>0}
\een
has the structure of a one-dimensional locally constant factorization algebra. 
Hence, $r_* \Obs_{z_0}$ is identified with a dg associative algebra. 
By work of Lurie and Francis-Ayala, this associative algebra is equivalent to the Hochschild homology
\ben
HH_*(\Obs_{z_0}) \simeq r_*(\Obs_{z_0}) .
\een 
The Hochschild homology of any $\sE_2$-algebra is an associative algebra.
Since pushforward and Hochschild homology is compatible with Koszul duality, we observe that the Theorem implies an equivalence of dg associative algebras
\ben
r_*(\Obs_{z_0}) \simeq HH_*(\Obs_{z_0}) \simeq HH_*(\Fin({Y_{\hbar} \fg}))
\een 
We have just mentioned the first quasi-isomorphism. 
The second one follows from the fact that the Hochschild homology of any $\sE_2$ algebra $A$ is quasi-isomorphic as $\sE_1$-algebras to the Hochschild homology of its category of perfect modules. 
Further, $\Perf(\Obs_{z_0}) \simeq \Fin(Y_\hbar \fg)$ by the main theorem. 

This gives us an explicit way of realizing representations of the Yangian inside of the four-dimensional gauge theory. 
We can rightfully view $r_*(\Obs_{z_0})$ as the value of the factorization algebra on a circle $\{z_0\} \times S^1_w$. 
Given a finite dimensional representation $V$ of $Y_\hbar \fg$ we can consider its trace
\ben
{\rm Tr}_V : Y_\hbar \fg \to \CC .
\een 
This trace is a character, hence determines an element ${\rm Tr}_V \in HH_0({\rm Fin}(Y_\hbar \fg))$. 
This theorem implies that characters for the Yangian appear as ``Wilson loops" in the four-dimensional gauge theory.

So far we have been primarily concerned with the factorization algebra in the topological $\RR^2_w$-direction. 
What extra structure does the factorization in the $\CC_z$ direction buy us?
Note that the theory is {\em not} locally constant in the $\CC_z$-direction. 
In fact, it is {\em holomorphic}, so the factorization structure actually induces a sort of operator product expansion (OPE) just as in the theory of vertex algebras. 

The way to codify this is to consider the operator product expansion between a small disk placed at $z = z_0$ and a small disk placed at $z= z_0 + \lambda$, where $\lambda \in \CC^\times$. 
By holomorphicity, we can write the OPE as
\ben
F_{OPE} : \Obs_{z_0} \tensor \Obs_{z_0} \to \Obs_{z_0} ((\lambda)),
\een
where now $\lambda$ plays a formal role. 
By the main theorem, we can identify this with a monoidal bifunctor of the form
\ben
F_{OPE} : \Fin(Y_\hbar \fg) \times \Fin (Y_{\hbar} \fg) \to \Fin(Y_\hbar \fg) ((\lambda)) .
\een 

\begin{thm} The map $F_{OPE}$ encodes Drinfeld's universal $R$-matrix 
\ben
R(\lambda) \in Y_\hbar(\fg) \tensor Y_{\hbar}(\fg) ((\lambda)) .
\een
Thus, the factorization product in the $\CC_z$-direction encodes the universal $R$-matrix.
\end{thm}

We can state this result in a more categorical framework as follows.
We have just mentioned that the theory spits out a factorization algebra {\em on} $\CC_z$ {\em with values in $\sE_2$-algebras}. 
As above, every $\sE_2$-algebra admits a module category that is monoidal.
In particular, we have a factorization algebra on $\CC$ with values in the monoidal category of modules. 
By the main theorem, this factorization algebra is equivalent to a factorization algebra on $\CC$ with values in the monoidal category of modules for $Y_\hbar \fg$. 
This witnesses a structure in the category $\Fin(Y_\hbar(\fg))$ that is not so obvious: in addition to being monoidal, it is also a {\em chiral category}.

To set up the analogy with Chern-Simons, consider the following. 
Placing Chern-Simons on the three-manifold $\Sigma \times \RR$ allows us to think about the theory as maps from $\RR$ to local systems on $\Sigma$. 
At the level of factorization algebras, in the case $\Sigma = \RR^2$, this says that we have an $\sE_1$-algebra with values in $\sE_2$-algebras.
If we look at modules for the $\sE_2$-algebra, we obtain a $\sE_1$-algebra in $\sE_1$-categories, so an $\sE_2$-category.

We think about the Yangian as being equivalent to maps from $\CC$ to the moduli of flat connections on $S = \RR^2$.
On $\CC$, the theory is holomorphic, so we obtain a chiral, or vertex algebra, with values in $\sE_2$-algebras. 
Looking at modules, we get a chiral algebra in $\sE_1$-categories. 
This is what we should call a ``chiral monoidal category". 
 
%\input{first section}
\bibliographystyle{alpha}
%\bibliographystyle{spmpsci}  
\bibliography{syllabus}

\end{document}