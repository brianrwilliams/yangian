\documentclass[10pt]{article}

\usepackage{macros}

%Formatting
\linespread{1.25}

\usepackage{parskip}
\setlength{\parindent}{18pt}
\setlength{\parindent}{0cm}

\setcounter{tocdepth}{2}
\setcounter{secnumdepth}{5}

\title{The Yangian and four-dimensional gauge theory}

%\author{Brian R. Williams}
\date{} 

\def\brian{\textcolor{blue}{BW: }\textcolor{blue}}

\begin{document}
\maketitle

\brian{Valerio, can you add that first paragraph about quantum groups? I've deleted mention of it below. Maybe something like: The type of infinite dimensional Lie algebras we study are of the form $\fg[[z]]$ where $z$ is a formal variable.....The {\em Yangian} deformations are a type of quantum group, and are realized as deformations of the universal enveloping algebra $U(\fg[[z]])$. These deformations have appeared in various mathematical contexts, most notably in the theory of integrable systems and statistical mechanics.}

The purpose of this course is to explore a relationship between a certain class of gauge theories defined on four-manifolds and deformations of infinite dimensional Lie algebras. 
Mathematically, a gauge theory involves studying connections defined on bundles over a smooth manifold. 
These theories are motivated from physics (Yang-Mills gauge theory), but often have neat and concise mathematical descriptions. 

The gauge theory we study is defined on manifolds of the form $\Sigma \times S$ where $\Sigma$ is Riemann surface and $S$ is a real two-dimensional manifold. 
Mathematically, the class of connections defining the gauge theory are, in a precise sense, {\em holomorphic} in the direction of $\Sigma$ and {\em flat} in the direction of $S$.  
The connection between gauge theory and quantum groups we will study is similar in spirit to perhaps a more well-studied relationship between Chern-Simons theory, the study of flat connections on three-manifolds, and quantum groups.

We follow the seminal work of Costello \cite{CosInt, CosYangian} showing how Yangians arise from the algebra of operators of the four-dimensional gauge theory.
The key to this result relies on the formalism of Costello-Gwilliam \cite{CG1,CG2} that the algebra of operators of a quantum field theory form a {\em factorization algebra}.  
This is a vast generalization of the description of algebras of operators in conformal field theory as vertex algebras. 
Factorization algebras simultaneously generalize the notion of a vertex algebra and algebras over more familiar operads, such as the operad of little disks. 
The primary goal of this seminar is to study Costello's construction which starts from the factorization algebra description of the operators of a four dimensional gauge theory and recovers the Yangian quantum group of the gauge Lie algebra. 
Time permitting, we can focus on more concrete and computational sides of the program began in the works \cite{CWY1, CWY2}.

{\bf Some keywords}: Factorization algebras, Koszul duality, quantum groups, Yangians, complex geometry. 

{\bf Informal prerequisites}: \ul{No background in physics is necessary!} Basic category theory, homological algebra, and rudiments of Lie algebras.
Knowledge of some basic differential geometry including vector bundles, connections, and differential forms. 

{\bf Organizers:} Chris Beasley, Valerio Toledano Laredo, Brian Williams

{\bf Time/Place} Tuesdays, 3:30 PM -- 7:30 PM 

\subsection*{Detailed (approximate) syllabus}

\begin{itemize}

\item[Week 1, Sep 11]

{\em Introduction.} Speaker: Brian 

\item[Week 2, Sep 18] 

{\em An introduction to factorization algebras}. Speaker: Ryan Mickler.

The definition of a factorization algebra with values in a symmetric monoidal category.
Lurie's result that locally constant factorization algebras on $\RR^n$ are equivalent to $E_n$-algebras. 
Hochschild homology as a special case of factorization homology.
\cite{CG1, LurieHA, AFTopMan}.


\item[Week 3, Sep 25]

{\em Koszul duality for $E_n$-algebras}.
Categories of (co)modules for $E_n$-algebras.
Koszul duality for associative algebras and its generalization for augmented $E_n$-algebras (with special attention to the case $n=2$).
Outline result of Tamarkin that the Koszul dual of an $E_2$-algebra is a Hopf algebra. 
Interplay between Hochschild homology and Koszul duality.
\cite{CosYangian, Tamarkin}.


\item[Week 4, Oct 2] {\em ``Four-dimensional Chern-Simons theory"}.
The moduli space of holomorphic, partially flat, connections on a complex surface.
%\footnote{Roughly, a holomorphic partially flat connecton on a complex manifold of the form $X \times Y$ is a holomorphic connection that is holomorphically flat in the $Y$-direction.}
The moduli space of multiplicative Higgs bundles on a Riemann surface.


\item[Week 5, Oct 9] {\em Batalin-Vilkovisky quantization and renormalization}.
An introduction to BV quantization for gauge theories.


\item[Week 6, Oct 16] {\em Batalin-Vilkovisky quantization and renormalization (cont.)}.

\item[Week 7, Oct 23] {\em Observables of the four-dimensional gauge theory}.
Deformation of functions on the classical moduli space defined by BV quantization to small orders in $\hbar$. 
Line operators and a generalized version of ``conformal blocks" from CFT. 
\cite{CosBook, CG2, CosYangian}

\item[Week 8, Oct 30] {\em Costello's main result}.
\begin{thm}
The Koszul dual of the $E_2$-algebra of quantum observables of four-dimensional Chern-Simons theory on $\CC_z \times \RR^2_w$ (restricted to a factorization algebra on $\{z=0\} \times \RR^2_w$) is Koszul dual to the Yangian Hopf algebra. 
\end{thm}
\cite{CosBook, CG2, CosYangian,CWY1,CWY2}

\item[Week 9, Nov 6] {\em Quantum groups and the Yangian}.
%\brian{Not sure exactly how to structure this, some input would be greatly appreciated!}
Drinfeld's universal $R$-matrix. 
Relationship to integrable systems and lattice models \cite{Etingof, ChariPressley}.


\item[Week 10, Nov 13] {\em Holomorphic factorization and the universal $R$-matrix}
Background on vertex algebras.
Describe functor from the category of holomorphic factorization algebras to vertex algebras.
Relationship to {\em chiral algebras} of Beilinson-Drinfeld. 
\cite{CG1, BD}. 
The quantum OPE as a map of $E_2$-algebras.
Hochschild homology for categories \cite{CosYangian}.

\item[Week 11, Dec 4] {\em The quantum Yang-Baxter equation}
How the quantum master equation for BV quantization implies the quantum Yang-Baxter equation.
\cite{CWY1,CWY2}.

\item[Week 12, Dec 11] {\em Enhancements and variations of the construction.}
Realizing various spin systems by tweaking the input data. 
Coupling to surface operators. 

\item[Weeks ??] 

\end{itemize} 

%\begin{itemize}
%\item An introduction to factorization algebras valued in symmetric monoidal $\infty$-categories. Lurie's result that locally constant factorization algebras recover algebras over the operad of little disks.
%Modules for $E_n$-algebras.\cite{CG1, LurieHA, AFTopMan}
%\item Holomorphic factorization algebras and their relationship to chiral and vertex algebras. \cite{CG1, BD}
%\item Koszul duality for $E_n$-algebras.
%Tamarkin's \cite{Tamarkin} result that the Koszul dual of an $E_2$ algebra is a Hopf algebra.
%The Hochschild homology of $E_n$ algebras.
%%\cite{FrancisAyala, ...}
%\item An introduction to quantum groups and the Yangian.
%Drinfeld's universal $R$-matrix. \cite{Etingof, ChariPressley}
%Relationship to integrable systems and lattice models.
%\item The classical moduli space of holomorphic partially flat connections on complex surfaces.\footnote{Roughly, a holomorphic partially flat connecton on a complex manifold of the form $X \times Y$ is a holomorphic connection that is holomorphically flat in the $Y$-direction.}
%The Batalin-Vilkovisky formalism and quantization.
%Costello-Gwilliam's result that the operators of a gauge theory (or any QFT) form a factorization algebra. \cite{CosBook, CG2, CosYangian}
%\item Line operators of the four dimensional gauge theory and their interpretation via Koszul duality.
%Possibly introduce Hochschild homology for monoidal categories. 
%\item Computing the universal $R$-matrix from the BV quantization. 
%\item Enhancements and variations of the construction. 
%Realizing various spin systems by tweaking the input data. 
%Coupling to surface operators. 
%\end{itemize}

%\tableofcontents

%\section*{Acknowledgements} 

%\input{first section}
\bibliographystyle{alpha}
%\bibliographystyle{spmpsci}  
\bibliography{syllabus}

\end{document}