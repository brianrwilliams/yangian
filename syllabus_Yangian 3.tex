\documentclass[10pt]{article}

\usepackage{macros}

%Formatting
\linespread{1.25}

\usepackage{parskip}
\setlength{\parindent}{18pt}
\setlength{\parindent}{0cm}

\setcounter{tocdepth}{2}
\setcounter{secnumdepth}{5}

\title{The Yangian and four dimensional gauge theory}

%\author{Brian R. Williams}

\def\brian{\textcolor{blue}{BW: }\textcolor{blue}}

\begin{document}
\maketitle

There is a deep connection between gauge theory and quantum groups.
Perhaps the most ubiquitous is the relationship between the category of line operators in Chern-Simons theory on three manifolds and the braided monoidal category of representations of the quantum group.
In the seminal work, Costello \cite{CosInt, CosYangian} has shown how certain infinite dimensional quantum groups, Yangians, arise from the algebra of operators of a {\em four} dimensional (supersymmetric) gauge theory.
The key to this result relies on the formalism of Costello-Gwilliam that the factorization algebras of a quantum field theory form a factorization algebra. 
This is an enhancement of the well-known description of a conformal field theory in terms of vertex, or chiral, algebras. 
The primary goal of this seminar is to outline Costello's construction which starts from the factorization algebra description of the observables of a four dimensional gauge theory and recovers the Yangian quantum group of the gauge Lie algebra. 
Time permitting, we can f ocus on more concrete and computational sides of the program began in the works \cite{CWY1, CWY2}.

{\bf Some keywords}: Factorization algebras, Koszul duality, quantum groups, Yangians. 

{\bf Informal prerequisites}: Basic category theory, homological algebra, and representation theory for Lie algebras.
Knowledge of vector bundles, connections, and differential forms. 

\begin{itemize}
\item[Week 1, Aug 27--31] 
{\em An introduction to factorization algebras}.  
The definition of a factorization algebra with values in a symmetric monoidal category.
Lurie's result that locally constant factorization algebras on $\RR^n$ are equivalent to $E_n$-algebras. 
Hochschild homology as a special case of factorization homology.
\cite{CG1, LurieHA, AFTopMan}.


\item[Week 2, Sep 4--7] 
{\em Holomorphic factorization algebras and vertex algebras}. 
Background on vertex algebras.
Describe functor from the category of holomorphic factorization algebras to vertex algebras.
Relationship to {\em chiral algebras} of Beilinson-Drinfeld. 
\cite{CG1, BD}. 

\item[Week 3, Sep 10--14]
{\em Koszul duality for $E_n$-algebras}.
Categories of (co)modules for $E_n$-algebras.
Koszul duality for associative algebras and its generalization for augmented $E_n$-algebras (with special attention to the case $n=2$).
Outline result of Tamarkin that the Koszul dual of an $E_2$-algebra is a Hopf algebra. 
Interplay between Hochschild homology and Koszul duality.
\cite{CosYangian, Tamarkin}.

\item[Week 4, Sep 17--21] 
{\em Quantum groups and the Yangian}.
\brian{Not sure exactly how to structure this, some input would be greatly appreciated!}
Drinfeld's universal $R$-matrix. 
Relationship to integrable systems and lattice models.
\cite{Etingof, ChariPressley}

\item[Week 5, Sep 24--28] {\em Quantum groups and the Yangian (cont.)}.

\item[Week 6, Oct 1--5] {\em ``Four-dimensional Chern-Simons theory"}.
The moduli space of holomorphic, partially flat, connections on a complex surface.\footnote{Roughly, a holomorphic partially flat connecton on a complex manifold of the form $X \times Y$ is a holomorphic connection that is holomorphically flat in the $Y$-direction.}
The moduli space of multiplicative Higgs bundles on a Riemann surface.

\item[Week 7, Oct 9--12] {\em Batalin-Vilkovisky quantization and renormalization}.
An introduction to BV quantization for gauge theories. \brian{More details}

\item[Week 8, Oct 15--19] {\em Batalin-Vilkovisky quantization and renormalization (cont.)}.

\item[Week 9, Oct 22--26] {\em Observables of the four-dimensional gauge theory}.
Deformation of functions on the classical moduli space defined by BV quantization to small orders in $\hbar$. 
Line operators and a generalized version of ``conformal blocks" from CFT. 
\cite{CosBook, CG2, CosYangian}

\item[Week 10, Oct 29--Nov 2] {\em Costello's main result}.
\begin{thm}
The Koszul dual of the $E_2$-algebra of quantum observables of four-dimensional Chern-Simons theory on $\CC_z \times \RR^2_w$ (restricted to a factorization algebra on $\{z=0\} \times \RR^2_w$) is Koszul dual to the Yangian Hopf algebra. 
\end{thm}
\cite{CosBook, CG2, CosYangian,CWY1,CWY2}

\item[Week 11, Nov 5--9] {\em Holomorphic factorization and the universal $R$-matrix}
The quantum OPE as a map of $E_2$-algebras.
Hochschild homology for categories.
\cite{CosYangian}.

\item[Week 12, Nov 13--16] {\em The quantum Yang-Baxter equation}
How the quantum master equation for BV quantization implies the quantum Yang-Baxter equation.
\cite{CWY1,CWY2}.

\item[Weeks ??] Enhancements and variations of the construction. 
Realizing various spin systems by tweaking the input data. 
Coupling to surface operators. 
\end{itemize} 

%\begin{itemize}
%\item An introduction to factorization algebras valued in symmetric monoidal $\infty$-categories. Lurie's result that locally constant factorization algebras recover algebras over the operad of little disks.
%Modules for $E_n$-algebras.\cite{CG1, LurieHA, AFTopMan}
%\item Holomorphic factorization algebras and their relationship to chiral and vertex algebras. \cite{CG1, BD}
%\item Koszul duality for $E_n$-algebras.
%Tamarkin's \cite{Tamarkin} result that the Koszul dual of an $E_2$ algebra is a Hopf algebra.
%The Hochschild homology of $E_n$ algebras.
%%\cite{FrancisAyala, ...}
%\item An introduction to quantum groups and the Yangian.
%Drinfeld's universal $R$-matrix. \cite{Etingof, ChariPressley}
%Relationship to integrable systems and lattice models.
%\item The classical moduli space of holomorphic partially flat connections on complex surfaces.\footnote{Roughly, a holomorphic partially flat connecton on a complex manifold of the form $X \times Y$ is a holomorphic connection that is holomorphically flat in the $Y$-direction.}
%The Batalin-Vilkovisky formalism and quantization.
%Costello-Gwilliam's result that the operators of a gauge theory (or any QFT) form a factorization algebra. \cite{CosBook, CG2, CosYangian}
%\item Line operators of the four dimensional gauge theory and their interpretation via Koszul duality.
%Possibly introduce Hochschild homology for monoidal categories. 
%\item Computing the universal $R$-matrix from the BV quantization. 
%\item Enhancements and variations of the construction. 
%Realizing various spin systems by tweaking the input data. 
%Coupling to surface operators. 
%\end{itemize}

%\tableofcontents

%\section*{Acknowledgements} 

%\input{first section}
\bibliographystyle{alpha}
%\bibliographystyle{spmpsci}  
\bibliography{syllabus}

\end{document}