\documentclass[11pt]{amsart}

\usepackage{macros}

%Formatting
\linespread{1.25}

\usepackage{parskip}
\setlength{\parindent}{18pt}
\setlength{\parindent}{0cm}

\setcounter{tocdepth}{2}
\setcounter{secnumdepth}{5}

\def\yscomod{{\rm Comod}_{Y^*(\fg)}}
\def\yslcomod{{\rm Comod}_{Y^*(\fg)}((\lambda))}

\title{$R$-matrices from $4$-dimensional factorization algebras}

\author{Brian R. Williams}
\date{} 

\def\brian{\textcolor{blue}{BW: }\textcolor{blue}}

\begin{document}
\maketitle

\section{A recollection of $R$-matrices}

\subsection{Warmup: braided monoidal categories}

Suppose that $\sC$ is an $\sE_2$, or braided, monoidal category.
We have seen that we can view an $\sE_2$ category as an $\sE_1$ object in $\sE_1$ monoidal (or just monoidal), categories
\[
\sC \in {\rm Alg}_{\sE_2} (\sC{\rm at}) = {\rm Alg}_{\sE_1} \left({\rm Alg}_{\sE_1} (\sC{\rm at}) \right) .
\]
Concretely, this means that on $\sC$ we have two {\em compatible} tensor products $\tensor, \boxtimes$.
Actually, $\tensor, \boxtimes$ are naturally isomorphic
\begin{align*}
V \boxtimes W & \cong (V \tensor 1) \boxtimes (1 \tensor W) \\ & \cong (V \boxtimes 1) \tensor (1 \boxtimes W) \\ & \cong V \tensor W .
\end{align*}

All of the juice, therefore, is in the requirement that $\tensor$ and $\boxtimes$ are compatible. 
The most efficient way to encode this is to require that the bifunctor
\[
\boxtimes : \sC \times \sC \to \sC
\]
determined by $\boxtimes$ is {\em monoidal}, where we view $\sC \times \sC, \sC$ as monoidal categories via $\tensor$. 
Spelling out what this means boils down to requiring that there is a natural isomorphism
\[
(V \tensor W) \tensor (V' \tensor W') \cong (V \tensor V') \tensor (W \tensor W')
\]
for all objects $V,V',W,W'$.
Setting $V, W'= 1$, we see that this isomorphism determines an isomorphism $\sigma_{W, V'} : W \tensor V' \to V' \tensor W$.
The resulting structure $(\sC, \tensor, \sigma_{V,W})$ defines a {\em braided} monoidal category, where $\sigma_{V,W}$ are the braiding morphisms.

Conversely, given a braided monoidal category we can define
\[
(V \tensor W) \tensor (V' \tensor W') \cong V \tensor (W \tensor V') \tensor W'  \overset{\sigma_{W, V'}}{\cong} V \tensor (V' \tensor W) \tensor W' \cong (V \tensor V') \tensor (W \tensor W') .
\]
In other words, a braided monoidal category determines a monoidal structure on the bifunctor $(V,W) \mapsto V \tensor W$, and vice-versa. 

\subsection{Drinfeld's rational $R$-matrix}
We have been working with topological algebras / coalgebras, and modules / comodules over them that have an additional topological structure. 
We arrived at this topological structure through the data of a filtration.

In Week ??, the category of filtered, bounded, free, dg $\CC[[\hbar]]$-modules $\sC^b$ was introduced. 
%The filtration weight of $\hbar$ is $+1$. 
An object $V \in \sC^b$ is a cochain complex of the form
\[
V_0 \tensor \CC[[\hbar]] 
\]
where $V_0$ is a cochain complex equipped with a decreasing filtration $$V = F^0 V \supset F^1 V \supset \cdots$$ and satisfies $V \cong \lim V / F^i V.$
The category $\sC^b$ is symmetric monoidal.
The tensor product of two objects $V, W \in \sC^b$ is the completed one, where we take into account the filtrations in the following way
\[
V \tensor_{\CC[[\hbar]]} W = \lim_{i,j} \left((V / F^i V) \tensor_{\CC[[\hbar]]} W / F^j W\right)  \in \sC^b .
\]

Introduce the formal Laurent series ring $\CC((\lambda))$. 
We will need to consider a Hopf algebra built from $Y(\fg)$ by adjoining the parameter $\lambda$. 
To do things properly, we need to do this in the following completed way.
For $V$ an object in $\sC^b$, define the completed tensor product
\[
V \Hat{\tensor} \CC((\lambda)) = \lim_i \underset{k}{\rm colim} \lim_j \lambda^{-k} (V / F^i V) [\lambda] / \lambda^j .
\]
We will often simply denote this by $V((\lambda))$. 

\begin{rmk}
This formula for the completed tensor product arises from viewing $\CC((\lambda))$ as an ind-pro object
\[
\CC((\lambda)) = \underset{k}{\rm colim} \; \lambda^{-k} \CC[[\lambda]] = \underset{k}{\rm colim} \lim_{j} \lambda^{-k} \CC[\lambda] / \lambda^j .
\]
\end{rmk}

\begin{rmk}
Throughout, we let $\tensor_{alg}$ denote the algebraic tensor product. 
Note that $Y(\fg)((\lambda))$ and $Y(\fg)\tensor_{alg} \CC((\lambda))$ are {\em different} Hopf algebras, but we have an inclusion of Hopf algebras
\[
Y(\fg)\tensor_{alg} \CC((\lambda)) \hookrightarrow Y(\fg)((\lambda)).
\] 
\end{rmk}

The category of $Y(\fg)((\lambda))$-modules is defined similarly as in in the plain $Y(\fg)$ case. 
In particular, we focus on modules that are quasi-free, which means they admit a filtration whose assoociated graded pieces are of the form $V \tensor_{alg} (A((\lambda)))$. 

Recall that $Y(\fg)$ is a topological Hopf algebra over the ring $\CC[[\hbar]]$. 
Moreover, there is a $\CC^\times$-action on $Y(\fg)$ inducing a grading where $\hbar$ has weight $1$. 
We define $\lambda$ to have $\CC^\times$-weight $1$.


\begin{thm}[\cite{Drin87}] \label{thm: drin1}
There is a unique element 
\[
R(\lambda) \in Y(\fg) \Hat{\tensor}_{\CC[[\hbar]]} Y(\fg) \Hat{\tensor} \CC((\lambda))
\]
satisfying:
\begin{enumerate}
\item[(1)]\label{invariance} ($\CC^\times$-invariance) $R(\lambda)$ is $\CC^\times$-invariant. 
\item[(2)]\label{equiv} (Translation equivariance)
For all $a \in Y(\fg)$ 
\[
(T_\lambda \tensor T_0) \sigma \Delta(a) = R(\lambda) ((T_\lambda \tensor T_0) \Delta(a)) R(\lambda)^{-1} \in R(\lambda) \in Y(\fg) \Hat{\tensor}_{\CC[[\hbar]]} Y(\fg) \Hat{\tensor} \CC((\lambda)) .
\]
\item[(3)] \label{R3} (??) $(\Delta \tensor 1) R(\lambda) = R_{13}(\lambda) R_{23}(\lambda)$ and $(1 \tensor \Delta) R(\lambda)$. 
\end{enumerate}
\end{thm}

\begin{rmk}
The $R$-matrix $R(\lambda)$ satisfies the Yang-Baxter equation with spectral parameter. 
This equation follows from the three listed properties in the theorem. 
\end{rmk}

\def\ymodfin{{\rm Mod}_{Y(g)}^{fin}}

\subsection{Interpreting the spectral $R$-matrix}

Our goal in this section is to interpret the $R$-matrix conditions in Theorem \ref{thm: drin1} in a more categorical way. 
The running example we should keep in mind from Section \ref{sec: braid} is that to give a braided monoidal structure is equivalent to prescribing monoidality of the tensor product bifunctor. 
We will see something similar in the case of the $R$-matrix: the data of a rational $R$-matrix is equivalent to the monoidality of a certain functor.
See Theorems \ref{thm: fun1} and \ref{thm: comodequiv}. 

For the remainder of this section, the category $\ymodfin$ denotes the category of (non dg) $Y(\fg)$-modules that are free and finite dimensional as $\CC[[\hbar]]$-modules. 

\begin{dfn}
If $V \in \ymodfin$, define the $Y(\fg)((\lambda))$-module
\[
T_\lambda V = V \tensor_{Y(\fg)} Y(\fg)((\lambda)) .
\]
In the definition, we use the Hopf algebra homomorphism $T_\lambda : Y(\fg) \to Y(\fg)((\lambda)) = Y(\fg) \Hat{\tensor} \CC((\lambda))$. 
Similarly, define the $Y(\fg)((\lambda))$-module
\[
T_0 V = V \Hat{\tensor} \CC((\lambda)) = V((\lambda))
\]
as the induction of $V$ along the embedding of Hopf algebras $T_0 : Y(\fg) \to Y(\fg)((\lambda)) \;, \; a \mapsto a \tensor 1$. 
\end{dfn}

With this notation in place, we can define the following functor that will be relevant for the remainder of the section. 
Define
\begin{equation}\label{functor}
F_\lambda : \ymodfin \times \ymodfin \to {\rm Mod}_{Y(\fg)}((\lambda))
\end{equation}
on objects by $(V, W) \mapsto T_0 V \tensor T_\lambda W$.
Similarly, we have the functor
\[
F'_\lambda : \ymodfin \times \ymodfin \to {\rm Mod}_{Y(\fg)}((\lambda))
\]
defined on objects by $(V, W) \mapsto T_\lambda V \tensor T_0 W$.

\begin{con}
Suppose $S(\lambda) \in Y(\fg) \tensor Y(\fg)((\lambda))$ is any element,
and fix $V, W \in \ymodfin$. 
Note that $V \tensor W = V \tensor_{\CC[[\hbar]]} W$ is a module for the Hopf algebra $Y(\fg) \tensor Y(\fg)$. 
Multiplication by $S(\lambda)$ on $V \tensor W ((\lambda))$ defines an endomorphism
\[
S_{V,W} \in {\rm End}_{\CC((\lambda))} (V \tensor W((\lambda))) .
\]
Note that for this endomorphism to be well-defined, we needed $V,W$ to be finite rank as $\CC[[\hbar]]$-modules.
\end{con}

\begin{lem}\label{lem: R1}
Suppose $S(\lambda) \in Y(\fg) \tensor Y(\fg)((\lambda))$ satisfies condition (3) of Theorem \ref{thm: drin1}. 
Then, $S(\lambda)$ defines a natural transformation of functors
\[
\eta_{S} : F_\lambda \xto{\cong} F'_\lambda \circ \sigma
\]
via the formula
\[
\eta_S (V,W) = \sigma \circ S_{V,W} : T_0 V \tensor T_\lambda W \to T_\lambda W \tensor T_0 V.
\]
\end{lem}

\begin{lem}
Suppose $S(\lambda) \in Y(\fg) \tensor Y(\fg)((\lambda))$satisfies condition (2) of Theorem \ref{thm: drin1}. 
Then, for all $V_1,V_2,V_3 \in \ymodfin$ the following diagrams commute 
%\label{Rd1}
%\label{Rd2}
\end{lem}

We need one last technical lemma before proceeding.
Suppose that $\sC, \sD$ are monoidal categories.
We view the tensor products $\tensor_{\sC}$, $\tensor_{\sD}$ as bifunctors
\[
\tensor_{\sC} : \sC \times \sC \to \sC \;\; , \;\; \tensor_{\sD} : \sD \times \sD \to \sD .
\]

Suppose that $G : \sC \to \sD$ a functor.
A natural transformation making $G$ a monoidal functor is the data of a natural transformation of functors $\sC \times \sC \to \sD$
\[
\eta : G \circ \tensor_{\sC} \xto{\cong} \tensor_{\sD} \circ (G \times G) .
\]
In other words, we have isomorphisms $G(c \tensor_{\sC} c') \cong G(c) \tensor_{\sD} G(c')$ that are natural in $c,c' \in \sC$. 

\begin{lem}\label{lem: monoidal1}
Let $F_\lambda$ be the functor from Equation (\ref{functor}).
Then, the data of a natural transformation making $F$ a monoidal functor is equivalent to the data of a natural transformation
\[
\eta : F_\lambda \xto{\cong} F'_\lambda \circ \sigma 
\]
such that the diagrams (\ref{Rd1}) and (\ref{Rd2}) commute. 
\end{lem}

We can summarize a consequence of the above lemmas in the following way. 

\begin{thm}\label{thm: fun1}
Drinfeld's $R$-matrix $R(\lambda)$ gives rise to a natural transformation $\eta_R$, as in Lemma \ref{lem: R1}, making the functor
\[
F_\lambda : \ymodfin \times \ymodfin \to {\rm Mod}_{Y(\fg)} ((\lambda)) .
\]
monoidal. 
\end{thm}

\subsection{A variation for comodules}

\subsubsection{An asymmetry between modules and comodules}
Thus far, we have used the fact that any $Y(\fg)$-module induces a $Y(\fg)((\lambda)) = Y(\fg) \Hat{\tensor} \CC((\lambda))$-module. 
In the comodule case, this does not work. 
The reason for this is the following. 

Suppose $A$ is any algebra in $\sC^b$. 
Then, we obtain a $\CC((\lambda))$ algebra $A((\lambda)) = A \Hat{\tensor} \CC((\lambda))$ as the following composition
\[
A((\lambda)) \tensor_{\CC((\lambda)), alg} A((\lambda)) \to A((\lambda)) \Hat{\tensor}_{\CC((\lambda))} A((\lambda)) = (A \tensor A) \Hat{\tensor} \CC((\lambda)) \xto{m_A} A \Hat{\tensor} \CC((\lambda)) = A((\lambda)) .
\] 
The first arrow is the embedding of the algebraic tensor product inside of the completed tensor product. 
This says that every algebra $A$ induces a $\CC((\lambda))$-linear algebra $A((\lambda))$.

The same is not true for coalgebras. 
Suppose $C$ is a coalgebra in $\sC^b$.
The coproduct extends $\CC((\lambda))$-linearly to a map
\[
C((\lambda)) \to (C \tensor C) ((\lambda)) = (C \tensor C) \Hat{\tensor} \CC((\lambda)) .
\] 
Note that the right hand side {\em differs} from the $\CC((\lambda))$-algebraic tensor product $C((\lambda)) \tensor_{\CC((\lambda)), alg} C((\lambda))$. 
There is only a map from this algebraic tensor product into the right hand side above.
Thus, $C$ does not induce a coalgebra structure on $C((\lambda))$. 
Moreover, this shows that the notion of cofree comodules does not exist for $C((\lambda))$. 

\subsubsection{}

Define 
\[
F^{\rm comod}_{\lambda} : \yscomod \times \yscomod \to \yslcomod 
\]
by $(V,W) \mapsto T_{\lambda} V \tensor T_0 W$. 
 
\begin{thm}\label{thm: comodequiv}
The following pieces of data are equivalent:
\begin{itemize}
\item[(i)] An $R$-matrix with spectral parameter
\[
R(\lambda) \in F^0(Y(\fg) \tensor Y(\fg))((\lambda))
\]
satisfying (1),(2), and (3) of Theorem \ref{thm: drin1}.
\item[(ii)] 
The data of a natural transformation making $F_\lambda$ a monoidal functor.
\end{itemize}
\end{thm}
\begin{proof}
By Lemma \ref{lem: monoidal1}, the data (ii) of making $F_\lambda$ monoidal is equivalent to a natural isomorphism
\[
\eta_{V,W} : T_0(V) \tensor T_\lambda(W) \cong T_\lambda(W) \tensor T_0(V),
\]
satisfying conditions \brian{from R matrix lemmas}. 
Thus, we must show that an $R$-matrix as in (i) is equivalent to this data.

Let $V, W$ be $Y^*(\fg)$ comodules \brian{what does this mean}, and let $\Delta_V, \Delta_W$ denote the coaction maps.
Any element $\alpha \in Y(\fg) \tensor Y(\fg)$ defines a composition
\[
V \tensor W \xto{\Delta_V \tensor \Delta_W} (Y^*(\fg) \tensor V) \tensor (Y^*(\fg) \tensor W) \xto{\alpha_{13}} V \tensor W .
\]

Since Drinfeld's $R(\lambda) \in Y(\fg) \tensor Y(\fg) ((\lambda))$ is $\CC^\times$-weight zero, it determines a filtration preserving map
\[
R_{V,W}(\lambda) : V \tensor W \to V
\]
\end{proof}

When such an $R(\lambda)$ is given, we will denote by $F_R$ or $F_{R(\lambda)}$ the associated monoidal functor. 

\section{The OPE}
\def\Obsmod{{\rm Mod}_{\Obs_0}}
\def\Obslmod{{\rm Mod}_{\Obs_0}((\lambda))}

\begin{prop}
The map of $\sE_2$ algebras arising from the factorization product of disks in the $z$-plane
\[
m : \Obs_{0} \tensor \Obs_{0} \to \Obs_{D(0,s)} \tensor \Omega^{0,*}(A)
\]
determines a map
\[
m_{OPE} : \Obs_0 \tensor \Obs_0 \to \Obs_0 \tensor \CC((z))
\]
that fits into the commuting square of $\sE_2$-algebras
\[
\begin{tikzcd}
\Obs_0 \tensor \Obs_0 \ar[r, "m_{OPE}"] \ar[d, "m"] & \Obs_0 \tensor \CC((z)) \ar[d] \\
\Obs_{D(0,s)} \tensor \Omega^{0,*}(D(0,r)) [z^{-1}] \ar[r] & \Obs_{D(0,s)} \tensor \CC((z)) .
\end{tikzcd}
\]
\end{prop}


\subsection{The main result}



\begin{thm}[\cite{CosYangian1}] 
There is a natural equivalence of monoidal functors
\[
F_{OPE} \cong F_{R(\lambda)} : \yscomod \times \yscomod \to \yslcomod
\]
where $F(R(\lambda))$ is the monoidal functor corresponding to Drinfeld's $R$-matrix as in Theorem \ref{thm: comodequiv}.
\end{thm}  
\begin{proof}
First, we need to show that there is an equivalence of functors (forgetting monoidality). 

Since $F_{OPE}$ is monoidal, there is an isomorphism
\[
F_{OPE} (V \times W) = F_{OPE}( (V \times 1) \tensor (1 \tensor W)) = F_{OPE} (V \times 1) \tensor F_{OPE}(1 \times W).
\]
Here, $1 = \CC[[\hbar]]$ is the trivial $\CC[[\hbar]]$-linear $Y(\fg)$-module.

It suffices to show that there is a natural quasi-isomorphism
\[
F_{OPE} (V \times 1) \tensor F_{OPE}(1 \times W) \cong T_0(V) \tensor T_\lambda(W) .
\]
Recall, $F_{OPE}$ was induced from the map of $E_2$ algebras
\[
m_{OPE}(\lambda) : \Obs_0 \tensor \Obs_0 \to \Obs_0((\lambda)) .
\]
When restricted to $\Obs_0 \tensor \{1\}$, this map is simply the $\CC((\lambda))$-linear extension of the identity map. 
Thus, $F_{OPE} (V \times 1) = T_0(V)$. 

We thus need to show that $F_{OPE} (1 \times W) = T_\lambda(W)$. 
For this, consider the derivation $\partial_z = \frac{\partial}{\partial z}$ acting on $\Obs_0$. 
Define
\[
\Tilde{T}_\lambda : \Obs_0 \to \Obs_0 [[\lambda]]
\]
via $\Tilde{T}_\lambda (O) = e^{\lambda \partial_z} O \in \Obs_0 [[\lambda]]$. 

\begin{lem}
The restriction of $m'_{OPE}$
\[
m'_{OPE} (1 \tensor -) : \Obs_0 \to \Obs_0((\lambda))
\]
agrees with $\Tilde{T}_{-\lambda}$. 
\end{lem}
\begin{proof}
This is almost by construction. 
By Proposition \ref{prop: ope}, this map factors as
\[
\Obs_0 \to \Omega^{0,*}(D(0,r))[z^{-1}] \to \Omega^{0,*}(\Hat{D}) \simeq \CC((z))
\]
for any $r > 0$.
The middle map is power series expansion, which lands in the Dolbeault complex of the formal disk.
Since we are identifying $z \leftrightarrow -\lambda$, the lemma follows.
\end{proof}

The lemma implies that the functor 
\[
F(1 \times -) : \Obsmod \to \Obslmod
\]
sends 
\[
V \mapsto \Tilde{T}_{-\lambda}(V) = V \tensor_{\Obs_0} \;_1\left(\Obs_0\right)_{-\lambda}
\]
Here, $_1\left(\Obs_0\right)_{-\lambda}$ denotes the $\Obs_0 - \Obs_0$ bimodule for which an element $O \in \Obs_0$ acts by itself on the left and by $\Tilde{T}_{-\lambda}(O)$ on the right. 

The main result of this seminar is that the Koszul dual of $\Obs_0$ is the dual Yangian $Y^*(\fg)$. 
This Koszul duality trades modules over $\Obs_0$ for comodules over $Y^*(\fg)$. 
The endofunctor $\Tilde{T}_{-\lambda}$ on modules for $\Obs_0$ induces an endofunctor on the category of comodules for $Y^*(\fg)$. 

In terms of $\lambda$, there is a sign flip under this duality. 
That is, if $W$ is a $Y^*(\fg)$-comodule, define $\Tilde{T}_\lambda(W)$ to be the $Y^*(\fg)$-comodule $W \tensor_{Y^*(\fg)} \;_1Y^*(\fg)_\lambda$. 

The claim is that the functor $\Tilde{T}_{-\lambda}$ on $\Obs_0$-modules intertwines with the functor $\Tilde{T}_{\lambda}$ on $Y^*(\fg)$-comodules under Koszul duality. 

What we have just seen is that when we view $F_{OPE}$ as a bifunctor
\[
F_{OPE} : \yscomod \times \yscomod \to \yslcomod
\]
that $F_{OPE} (1 \times W) \simeq \Tilde{T}_\lambda(W)$. 
The last thing we need to show is that $\Tilde{T}_\lambda$ and $T_{\lambda}$ agree. 
The result follows from the following lemma.

\begin{lem}
There is a unique homomorphism of Hopf algebras $T : Y(\fg) \to Y(\fg) [[\lambda]]$
 such that
\begin{enumerate}
\item[(1)] 
Classically, $T|_{\hbar = 0} : U(\fg[[z]]) \to U(\fg[[z]])[[\lambda]]$ arises from the map of Lie algebras
\[
\fg[[z]] \to \fg[[z,\lambda]] \;\;\; , \;\;\; z \mapsto z + \lambda .
\]
\item[(2)] $T$ is $\CC^\times$-equivariant.
\item[(3)] $T \mod \lambda = {\rm id}$. 
\end{enumerate}
\end{lem}

From this lemma, the result is immediate.
We have just shown that $F_{OPE}(V \times W) \cong T_0(V) \tensor T_{\lambda}(W)$. 
Thus, $F_{OPE}$ agrees with $F_{R(\lambda)}$.

\end{proof}

\end{document}