\documentclass[11pt]{amsart}

\usepackage{macros}

%Formatting
\linespread{1.25}

\usepackage{parskip}
\setlength{\parindent}{18pt}
\setlength{\parindent}{0cm}

\setcounter{tocdepth}{2}
\setcounter{secnumdepth}{5}

\title{Quantum $4d$ gauge theory and an outline of the proof}

\author{Brian R. Williams}
\date{} 

\def\brian{\textcolor{blue}{BW: }\textcolor{blue}}

\begin{document}
\maketitle
 

\section{Existence (and uniqueness) of the quantum gauge theory}

We now turn to proving the existence and uniqueness of the quantization of the $4$-dimensional gauge theory. 
The result will hold for the ``generalized" Yangian theory on any complex surface $X$.
Part of the data of this generalized theory is a reducible divisor $D$ and a closed one-form. 
We recall the precise definition below. 

In this section we prove two results.

\begin{thm}\label{thm: bv quantization}
Let $X$ be a complex surface equipped with a holomorphic volume form. 
\begin{enumerate}
\item Holomorphic BF theory admits a unique quantization, compatible with certain symmetries we state below. 
\item If, in addition, we fix a reducible divisor $D$ and closed holomorphic one-form, the associated generalized Yangian theory admits a unique quantization, compatible with certain symmetries. 
\end{enumerate}
\end{thm}

We will prove existence and uniqueness by a direct cohomological calculation of the deformation complex. 
As we have seen, the Yangian theory is a deformation of holomorphic BF theory.
First we will prove (1), then study how the deformation affects our answer. 

Suppose $\sL$ is any local Lie algebra on $X$. 
Then, the {\em local deformation complex} is, by definition
\beqn\label{def}
{\rm Def}_{\sL} = \Omega^{top}_X \tensor_{D_X} \hbar \cred^*(J \sL) [[\hbar]] .
\eeqn
The idea for the proof of each of these claims is to consider the dg $D_X$-module $\cred^*(J \sL)$. 
We will show, if we look at invariants for certain natural symmetries, that this $D_X$-module is homotopically trivial in the degrees controlling obstructions and deformations. 
Thus, existence and uniqueness will follow. 

\subsubsection{}

Holomorphic BF theory makes sense for any bundle of Lie algebras. 
It is simply the cotangent theory to the moduli of holomorphic connections valued in the Lie algebra.
For $P$ a principal $G$-bundle, we can consider the adjoint bundle ${\rm ad}(P)$.
If, in addition, $D$ is a reduced divisor, we can consider the sheaf of Lie algebras ${\rm ad}(P)(-D) = {\rm ad}(P) \tensor \sO(-D)$. 
We study holomorphic BF theory with values in this sheaf of Lie algebras.
Let $\epsilon$ is an odd variable of degree $+1$. 
The local Lie algebra underlying this theory is
\[
\sL_{P,D} = \Omega^{0,*}(X , \fg_P(-D)) [\epsilon],
\]
equipped with the $\dbar$ operator and obvious Lie bracket. 

If we write the fields as $\alpha + \epsilon \beta$, the action functional is of the form
\[
\int_X \omega \wedge \left(\beta \wedge \dbar \alpha + \frac{1}{2} \beta \wedge [\alpha,\alpha]\right) ,
\] 
where $\omega$ is the holomorphic volume form. 

The first symmetry we consider is obtained by the operator $\frac{\partial}{\partial \epsilon}$. 
Of course, this operator commutes with $\dbar$. 
Moreover, the operator preserves the Lie bracket as well:
\[
\frac{\partial}{\partial \epsilon} [\alpha + \epsilon \beta, \alpha' + \epsilon \beta'] = \frac{\partial}{\partial \epsilon} \left(\epsilon [\alpha, \beta'] + \epsilon[\beta, \alpha']\right) = [\frac{\partial}{\partial \epsilon} (\alpha + \epsilon \beta), \alpha' + \epsilon \beta'] \pm [\alpha + \epsilon \beta, \frac{\partial}{\partial \epsilon} (\alpha' + \epsilon \beta')] .
\]
Thus, the one-dimensional abelian Lie algebra $\CC \cdot \frac{\partial}{\partial \epsilon}$ acts on $\sL_{P,D}$. 

There is also the action of the group, $\CC^\times$, that we'd like to take into account. 
We assign...

Suppose now that $\fh$ is {\em any} other Lie algebra acting on $\sL_{P,D}$. 
We assume that $\fh$ has the form $\fh = \fh_0 [\epsilon]$, so that $\CC \cdot \frac{\partial}{\partial \epsilon}$ acts on $\fh$. 
We look at deformations that are invariant with respect to the semi-direct product
\[
\CC \cdot \frac{\partial}{\partial \epsilon} \ltimes \fh = \CC \cdot \frac{\partial}{\partial \epsilon} \ltimes \fh_0 [\epsilon] ,
\]
with the additional property that the deformations are $\CC^\times$-equivariant. 

\begin{rmk}
Really, we are taking the {\em derived} invariants.
Recall, if $M$ is a $\fk$-module, then the derived invariants are given by the Chevalley-Eilenberg cochains $\clie^*(\fk, M)$. 
\end{rmk}

According to Equation \ref{def}, such deformations are equal to the $\CC^\times_{\rm cot}$ invariants of
\beqn\label{def bf}
{\rm Def}_{\sL} = \Omega^{top}_X \tensor_{D_X} \clie^*\left(\CC \cdot \frac{\partial}{\partial \epsilon} \ltimes \fh, \cred^*(J \sL) \right) .
\eeqn

\begin{lem}
The $\CC^\times_{\rm cot}$ invariants of the cohomology of (\ref{def bf}) vanishes in every degree. 
\end{lem}
\begin{proof}
Introduce the $D_X$-module
\[
M = \clie^*\left(\CC \cdot \frac{\partial}{\partial \epsilon} \ltimes \fh, \cred^*(J \sL) \right). 
\]
It suffices to show that the $\CC^{\times}_{\rm cot}$-invariants of the cohomology $D_X$-module of $M \tensor \hbar \CC[[\hbar]]$ vanish. 

Notice that $\sL$ is just the Dolbeault resolution of the holomorphic vector bundle ${\rm ad}(P)(-D)$. 
Thus
\[
J \sL \simeq J\left({\rm holomorphic\;sections\;of\;}\fg_P(-D)\right) .
\]

It follows that the stalk of $M$ at a point $x$ away from the divisor is quasi-isomorphic to
\[
M_x \simeq \clie^*\left(\CC \cdot \frac{\partial}{\partial \epsilon} \ltimes \fh, \cred^*(\fg[[z_1,z_2]][\epsilon]) \right) .
\]
Similarly, near the divisor the stalks look like
\[
M_x \simeq \clie^*\left(\CC \cdot \frac{\partial}{\partial \epsilon} \ltimes \fh, \cred^*(f\fg[[z_1,z_2]][\epsilon]) \right)
\]
where $f$ generates the ideal $\sO(-D)$. 

Since $\fh = \fh_0[\epsilon]$, note that in each case, we can write these stalks in the form
\beqn\label{def1}
 \clie^*\left(\CC \cdot \frac{\partial}{\partial \epsilon}, \clie^*(\fh_0, \cred^*(\fa [\epsilon])) \right) ,
\eeqn
for some Lie algebra $\fa$ on which $\fh_0$ acts. 

Note that $\hbar$ has $\CC^\times_{\rm cot}$ weight equal to $+1$. 
Since we really want the $\CC^\times_{\rm cot}$-invariants of $M \tensor \hbar \CC[[\hbar]]$, we only need to consider positive $\CC^\times_{\rm cot}$-weight spaces of (\ref{def1}). 

\begin{lem}
The cohomology of the positive weight spaces of (\ref{def1}) vanishes in every degree. 
\end{lem} 

Thus, we have concluded that each stalk is homotopy contractible, thus the $D_X$-module $M$ is also contractible and the result follows. 
\end{proof}

As a corollary of the calculations of this section, combined with the main result of Costello-Gwilliam in \cite{CG2}, we obtain the following:

\begin{thm}\label{thm: qfact}
Let $X$ be a complex surface equipped with a holomorphic volume form.
Then:
\begin{enumerate}
\item[(i)]
there is a factorization algebra $\Obs^{q}_{BF}$ on $X$ defined over $\CC[[\hbar]]$ such that
\[
\Obs^q_{BF} / \hbar \simeq \Obs^{cl}_{BF},
\] 
where $\Obs^{cl}_{BF}$ are the classical observables of holomorphic BF theory on $X$. ;
\item[(ii)] 
there is a factorization algebra $\Obs_{P,D}^q$
\end{enumerate}
\end{thm}

\section{From factorization algebras to Hopf algebras}

We now move towards proving one of the main results of this seminar: the Yangian quantum group is encoded in the factorization algebra associated to the quantization of the $4$-dimensional gauge theory. 
We make this statement precise in Theorem \cite{}. 

The main idea is to consider the $4$-dimensional theory on the manifold $\CC_z \times \RR^2_w$. 
For each point $z_0 \in \CC_z$ we consider the restriction of the resulting factorization algebra of quantum observables to $\{z_0\} \times \RR^2$. 
This factorization algebra is {\em locally constant} and hence corresponds, under Lurie's theorem, to an $E_2$-algebra. 
We will show that the Koszul dual of this algebra is equivalent to the Yangian, as {\em Hopf algebras}. 

\subsection{The quantum $E_2$ algebra}

First, we recall the {\em classical observables} of the $4$-dimensional theory on $\CC_z \times \RR^2$. 
It is easiest to write down once we fix a complex structure on $\RR^2 \cong \CC_w$. 
The elliptic complex defining the classical theory is
\[
\sL = \Omega^{0,*}(\CC^2 , \fg) \xto{\frac{\partial}{\partial w}} \Omega^{0,*}(\CC^2, \fg)[-1] .
\] 
The Lie bracket for $\fg$ gives $\sL$ the structure of a local dg Lie algebra on $X$. 
Moreover, the pairing
\[
\int \<\alpha, \beta\>_\fg \d z \d w
\]
equips $\sL$ is non-degenerate and degree $(-3)$. 
Hence, $\sL$ defines a classical theory in the BV formalism. 
This is precisely the ``classical Yangian theory" on $\CC_z \times \RR^2_w$ we've introduced in previous weeks. 

The classical observables $\Obs^{cl}_{4d}$ is the precosheaf on $\CC^2$:
\[
\Obs^{cl}_{4d} : U \subset \CC^2 \mapsto \clie^*(\sL(U)) = \left(\Sym \left(\sL(U)^\vee[-1]\right), \d_{CE}\right) 
\]
where $U$ is any open set. 
As always, by $\sL(U)^\vee$ we mean the topoloigical dual, and all tensor products are completed. 
By general principles, from Week 2, we know $\Obs^{cl}_{4d}$ has the structure of a factorization algebra on $\CC^2$. 

By Theorem \ref{thm: qfact}, the general theory of \cite{CG2} produces a factorization algebra $\Obs^{q}_{4d}$ linear over $\CC[[\hbar]]$, whose $\hbar \to 0$ limit is precisely $\Obs^{cl}_{4d}$. 
In the notation of Theorem \ref{thm: qfact}, here $P$ is the trivial $G$-bundle and $D = 0$. 

\subsubsection{}

We now discuss how to ``restrict" the factorization algebra to the $2$-dimensional submanifold $\{z_0\} \subset \CC_z \times \CC_w$.
Naively, a factorization algebra is only defined on open subsets, so we must appeal to a refined construction. 
The idea is to define the factorization algebra associated to a ``formal disk" in the $z$-direction. 

First, note that just like sheaves, factorization algebras pushforward.
If $f : X \to Y$ is smooth, and $\sF$ is a factorization algebra on $X$ then $f_* \sF$ assigns the open set $V \subset Y$ the complex $\sF(\pi^{-1}(V))$. 

Let $U \subset \CC_z$ be any open subset.
Of course, it makes sense to restrict $\Obs^{q}_{4d}$ to the open set $U \times \CC_w$. 
Consider the smooth map $\pi : U \times \CC_w \to \CC_w$. 
Pushing forward along $\pi$, we obtain a factorization algebra $\pi_*\left(\Obs^q_{4d}|_{U \times \RR^2}\right)$ on $\CC_w$. 

\begin{lem}
The factorization algebra $\pi_*\left(\Obs^q_{4d}|_{U \times \RR^2}\right)$ on $\CC_w$ is locally constant. 
\end{lem}
\begin{proof}
First, we show that this is true at the classical level. 
To any open set $V \subset \CC_w$, the pushforward factorization algebra assigns the complex
\[
\pi_*\left(\Obs^{cl}_{4d}|_{U \times \RR^2}\right) (V) = \Obs^{cl}_{4d}(U \times V) = \clie^*(\sL(U \times V)) .
\]
We must show that if $D \hookrightarrow D'$ is an inclusion of disks in $\CC_w$ that the induced factorization structure map is a quasi-isomorphism. 

For this, it suffices to show that at the level of sheaves of dg Lie algebras. 
That is, the inclusion $D \hookrightarrow D'$, the induced map $\sL(D') \to \sL(D)$ is a quasi-isomorphism of dg Lie algebas. 

If $D \subset \CC_w$ is any disk, note that there is a quasi-isomorphism 
\[
\sO^{hol}(U \times D) \xto{\simeq} \Omega^{0,*}(U \times D) .
\]
Thus, we have a quasi-isomorphism of dg Lie algebras 
\[
 \left (\sO^{hol}(U \times D) \tensor \fg \xto{\frac{\partial}{\partial w}} \sO^{hol}(U \times D) \tensor \fg [-1] \right) \xto{\simeq} \sL(U \times D) .
 \]
The differential $\partial / \partial w$ simply turns on the full de Rham differential, hence by the real Poincar\'{e} lemma for $D$, there is a quasi-isomorphism
\[
\sO^{hol}(U) \tensor \fg \xto{\simeq} \sL(U \times D) .
\]
These quasi-isomorphisms are compatible with inclusions, hence the classical factorization algebra is locally constant.
In fact, we have shown something stronger.
\begin{lem}
The classical factorization algebra $\pi_*\left(\Obs^{cl}_{4d}|_{U \times \RR^2}\right)$ is equivalent, as an $E_2$-algebra, to the commutative dg algebra
\[
\clie^*({\rm Hol}(U)\tensor \fg) .
\] 
\end{lem}

Turning to the quantum factorization algebra, note that there is a filtration on $\Obs^q_{4d}$ by powers of $\hbar$. 
This induces a spectral sequence (of factorization algebras) whose $E_1$ page is equal to $\Obs^{cl}_{4d} \tensor \CC[[\hbar]]$. 
The same is true for the pushforward factorization algebra. 
Since the classical factorization algebra is locally constant, it follows that the quantum one is as well. 

\end{proof}

\begin{rmk}
Instead of fixing an open set in $\CC_z$, this makes sense as we vary $U \subset \CC_z$.
So, we have shown that $\Obs^{q}_{4d}$ determines a factorization algebra on $\CC_z$ with values in $E_2$-algebras:
\[
\Obs^q_{4d} \in {\rm Fact}_{\CC_z}\left({\rm Alg}_{E_2}\right) .
\]
Note that the factorization algebra is {\em not} locally constant in the $z$-direction, so this is not equivalent to an $E_4$-algebra.
\end{rmk}

We take the open set $U \subset \CC_z$ to be a disk $U = D(z_0,r)$. 
Then, we have just shown that $\pi_*\left(\Obs^{cl}_{4d}|_{U \times \RR^2}\right) \simeq \clie^*(\sO^{hol}(D(z_0,r) \tensor \fg)$. 
Roughly, we consider a subspace of this factorization algebra induced by the power series expansion $\sO^{hol}(D(z_0,r)) \to \CC[[z-z_0]]$. 

\begin{lem}
There is an action by the group $S^1$ on $\Obs^{q}_{4d}$ which lifts the action of $S^1$ on $\CC_z \times \CC_w$ by rotations in the $z$-plane.
\end{lem}
\begin{proof}
The action by $S^1$ on $\CC_z \times \CC_w$ induces an action on the Dolbeualt complex and hence on $\sL$ as well. 
By our calculation of the deformation complex in the proof of Theorem \ref{thm: bv quantization} this action lifts to the quantum theory. 
\end{proof}

\begin{rmk}
Note that under this action by $S^1$ the symplectic pairing defining the classical theory has weight $+1$. 
Thus, the BV Laplacian defining the quantum theory has weight $-1$. 
This implies that $\hbar$ must have weight $+1$ under this action.
\end{rmk}

\begin{dfn}
Let $\pi : D(z_0,r) \times \CC_w \to \CC_w$ be projection.
Define the sub-factorization algebra on $\CC_w$:
\[
\Obs^{q,(k)}_{z_0} \subset \pi_*\left(\Obs^{q}_{4d}|_{D(z_0,r) \times \CC_w}\right)
\]
to be the weight $k$-eigenspace for the $S^1$ action. 
Let
\[
\Obs^q_{z_0} := \bigoplus_{k \in \ZZ} \Obs^{q,(k)}. 
\] 
\end{dfn}

Similarly, there is a classical version $\Obs^{cl}_{z_0}$ and a map of $E_2$ algebras
\[
\Obs^{q}_{z_0} \xto{\hbar \to 0} \Obs^{cl}_{z_0} .
\]
Moreover, this map is surjective, so that $\Obs^{q}_{z_0} = \Obs^{cl}_{z_0} [[\hbar]]$ as a graded (no differential) $\CC[[\hbar]]$-module. 
 
Note that $\Obs^{q}_{z_0}$ is a graded $E_2$-algebra (valued in cochain complexes) defined over the graded ring $\CC[[\hbar]]$. 
There is both the ``$S^1$ grading" and the intrinsic cochain complex grading, when we need to decipher between the two, we refer to the $S^1$ grading as the ``conformal dimension".

As a last check, note that as commutative dg algebras there is a quasi-isomorphism
\[
\Obs^{cl}_{z_0} \simeq \clie^*(\fg[[z-z_0]]) .
\] 

\subsection{Augmentation}

The crux of the seminar is that the observables of the $4d$ gauge theory are {\em Koszul dual} to the Yangian. 
Koszul duality makes sense in a wide variety of contexts, but one underlying theme persists: duality is defined for augmented algebras. 
Thus, in order to begin studying Koszul duality for the quantum factorization algebra, we must first produce an augmentation. 
This is the one point in the argument where we will utitlize the formulation of the generalized Yangian theory on manifolds other than $X = \CC^2 = \CC \times \RR^2$. 

By Theorem \cite{thm: bv quantization} we can define the quantum theory (in a unique way!) on any complex surface $X$ equipped with a holomorphic volume form, principal bundle, and reduced divisor. 
We consider the theory defined on
\[
X = \PP^1_z \times \CC_w
\]
where the volume form is $\d z \d w$ and the divisor is $\infty \times \CC_w$. 
Thus, through quantization, we get a factorization algebra $\Obs^{q}_{\PP^1 \times \CC_w}$ on $\PP^1 \times \CC_w$. 

We take the projection $\pi_{\PP^1} : \PP^1 \times \CC_w \to \CC_w$ and the resulting pushforward factorization algebra $\Obs^q_{\PP^1} := \pi_{\PP^1 *} \Obs^q_{\PP^1 \times \CC_w}$ on $\CC_w$. 
Similar methods as the previous section show that $\Obs^q_{\PP^1}$ is locally constant, hence defines and $E_2$ algebra. 

\begin{lem}
There is a quasi-isomorphism of $E_2$ algebras $\Obs^q_{\PP^1} \simeq \CC[[\hbar]]$. 
\end{lem}
\begin{proof}
Again, we first work classically. 
Note for any disk $D \subset \CC_w$ that
\[
\Obs^q_{\PP^1}(D) = \clie\left(\Omega^{0,*}(\PP^1, \sO(-\infty)) \tensor \Omega^*(D)\right) .
\]  
Since $H^*(\PP^1 , \sO(-\infty)) = 0$, the result follows. 
The usual spectral sequence argument completes the proof of the lemma. 
\end{proof}

Now, consider the diagram
\[
\begin{tikzcd}
\CC_{z} \times \CC_w \ar[dr, "\pi"'] \ar[rr,hook] & & \PP^1 \times \CC_w \ar[dl, "\pi_{\PP^1}"] \\
& \CC_w &
\end{tikzcd} 
\]
where we embed $\CC \hookrightarrow \PP^1$ away from $\infty$. 

It is immediate from our definitions that the generalized Yangian theory on $\PP^1 \times \CC_w$ restricts to the ordinary Yangian theory on $\CC_z \times \CC_w$.
Thus, we obtain a map of factorization algebras
\[
\Obs^{q}_{4d} \to \Obs^{q}_{\PP^1 \times \RR^2} .
\]

In particular, we obtain a sequence of $E_2$ algebras (= locally constant factorization algebras on $\CC_w$):
\[
\Obs_{z_0}^q \hookrightarrow \pi_*\left(\Obs^{q}_{4d}|_{D(z_0,r) \times \CC_w}\right) \to \Obs^q_{\PP_1} \simeq \CC[[\hbar]] .
\] 
The first map is just the inclusion of the $S^1$-eigenspaces. 
The second map follows from functoriality of pushforward.
The composition 
\beqn\label{aug}
\epsilon_{\hbar} : \Obs_{z_0}^q \to \CC[[\hbar]]
\eeqn
is the desired augmentation.

Summarizing:

\begin{cor}
BV quantization produces a map of $E_2$-algebras $\epsilon_{\hbar} : \Obs^q_{z_0} \to \CC[[\hbar]]$. 
Moreover, modulo $\hbar$, this augmentation
\[
\epsilon_\hbar \mod \hbar : \Obs^{cl}_{z_0} \simeq \clie^*(\fg[[z-z_0]]) \to \CC
\]
agrees with the standard one for Lie algebra cochains. 
\end{cor}

\subsection{Filtrations}

We will use a version of Koszul duality that is sensitive to a {\em filtration} on an algebra. 

In the case that $A = \Wedge^* V[-1]$, with the filtration by $F^i = \Wedge^{\geq i} V[-1]$, the filtered Koszul dual will still be $\Sym^*(V^\vee)$. 
In the case that $\fg$ is any dg Lie algebra and $A = \clie^*(\fg)$ with the filtration $F^i A = \Sym^{\geq i} (\fg)$, then $A^! \simeq U(\fg)^\vee$. 

In general, we will apply this filtered Koszul duality to the $E_2$ algebra we have just extracted from the factorization algebra of the $4d$ gauge theory. 

We introduce the filtration $\Obs^q_{z_0}$ that we will use in order to construct the filtered Koszul dual coalgebra. 
%The filtration is related to the $S^1$ weight grading that we have already discussed. 

%In general, consider an algebra $A$ defined over $\CC[[\hbar]]$. 
%Moreover, suppose $A = \oplus_{k \in \ZZ} A_k$ is a graded algebra where $\hbar$ has weight $+1$ under the grading. 
%This means that $\hbar^n A^{k-n} \subset A^{k}$, for instance. 
%Moreover, $A$...


\begin{dfn}
Recall, for any open $U \subset \RR^2_{w}$ the $S^1$ weight $k$ subspace $\Obs^{q,(k)}_{z_0}(U) \subset \Obs^{q}_{z_0}(U)$ on $\RR^2$. 
Define, for each $k$, the filtration on $\Obs^{q,(k)}_{z_0}(U)$ via
\[
F^j \Obs^{q,(k)}_{z_0} (U) = \oplus_{2m + n \geq j} \Sym^{n} \tensor \CC \cdot \hbar^m .
\]
The filtration is constructed so that it is compatible with the quantum differential $\d^\q F^j \subset F^j$.
\end{dfn}

\begin{rmk} In the definition of the factorization algebra $\Obs^q_{z_0}$, the direct sum $\oplus_k \Obs^{q,(k)}_{z_0}$ must be taken as a filtered coproduct. 
In this way, $\Obs^{q}_{z_0}$ is a factorization algebra with values in the category of filtered cochain complexes. 
Thus, $\Obs^{q}_{z_0}$ is an $E_2$ algebra in filtered cochain complexes. 
\end{rmk}

In particular, the classical limit $\Obs^{cl}_{z_0}$ is also filtered. (The dequantization map is a map of filtered $E_2$-algebras). 
It is immediate to see that modulo $\hbar$ this induces the usual filtration on $\Obs^{cl}_{z_0} \simeq \clie^*(\fg[[z-z_0]])$. 

\section{Sketch of the main result} 

We assume, for a moment, that the theory of filtered Koszul duality has been set up. 
We will address this issue in the next lecture. 
In particular, the filtered augmented $E_2$-algebra $\Obs^q_{z_0}$ admits a Koszul dual. 
To construct it, we forget the $E_2$-algebra structure to a filtered $E_1$ algebra structure then take the dual $(\Obs^q_{z_0})^{!}$ with respect to the augmentation $\epsilon_{\hbar}$ from Equation \ref{aug}. 

We will show that the remaining product endows $(\Obs^q_{z_0})^{!}$ with the structure of a bialgebra. 
In fact:

\begin{prop}
The bialgebra $(\Obs^q_{z_0})^{!}$ is a filtered Hopf algebra. 
\end{prop}

We will see this from a Tannakian formalism argument and is a general fact about taking the Koszul dual of fitlered $E_2$ algebras that are $E_2$ augmented. 

The first object we wish to compare to is the Yangian, or really, its dual. 
We will meet the rigorous definition of the Yangian quantum group $Y(\fg)$ soon, but for now we note that it is a topological Hopf algebra defined over $\CC[[\hbar]]$ such that modulo $\hbar$ there is an equivalence
\[
Y(\fg) / \hbar \cong U(\fg[[z]]) .
\] 

Also relevant to us is the {\em dual} Yangian. 
Being a Hopf algebra over $\CC[[\hbar]]$ we obtain the dual Hopf algebra via
\[
Y^*(\fg) := {\rm Hom}_{\CC[[\hbar]]} (\CC[[\hbar]], Y(\fg) .
\] 
Since the completed projective tensor product is compatible with topological duals, the Yangian is completely determined by the dual Yangian. 

We can finally state the main theorem. 

\begin{thm}
Let $\fg$ be a simple Lie algebra. 
Then, there is an isomorphism of $S^1$-equivariant topological Hopf algebras
\[
Y^*(\fg) \cong H^*(\Obs^q_{z_0})^{!} .
\] 
\end{thm}

The steps are as follows:
\begin{enumerate}
\item Verify, modulo $\hbar$, that there is an isomorphism of Hopf algebras
\[
H^*(\Obs^q_{0})^{!} \cong U(\fg[[z]])^\vee .
\] 

\item Modulo $\hbar^2$ the BV quantization agrees with the modulo $\hbar^2$ behavior of the Yangian. 

\end{enumerate}

Actually, (1) is easy. 
We've already seen that there is a filtered quasi-isomorphism of $E_2$ algebras
\[
\Obs_{0}^{q} / \hbar = \Obs^{cl}_{0} \simeq \clie^*(\fg[[z]]) .
\] 
Consequently, we can read off the filtered Koszul dual as
\[
H^*(\Obs^q_{z_0})^{!} \cong U(\fg[[z]])^\vee
\]
as desired. 
The thing to check is that this is an isomorphism of Hopf algebras, which will be the content of next weeks lectures. 

Step (2) is where we finally get our hands dirty and use the quantization machinery developed in the previous lectures. 
To state (2) more precisely, introduce the linear dual of the Koszul dual to the observables
\[
Y_{QFT}(\fg) = \left(\Obs^q_{z_0})^{!}\right)^\vee .
\] 
This is a Hopf algebra over $\CC[[\hbar]]$ such that modulo $\hbar$ 
\[
Y_{QFT} (\fg) / \hbar \simeq U(\fg[[z]])
\]
as Hopf algebas. 
Thus, it is the same as the Yangian $Y\fg$ modulo $\hbar$. 

The coalgebra structure on $Y_{QFT}(\fg)$ is of the form
\[
\Delta_{QFT} = \Delta^0 + \hbar \Delta_{QFT}^1 + \cdots  .
\]
Modulo $\hbar$, of course, $\delta_0$ determines a cocommutative coalgebra. 
Modulo $\hbar^2$, there is an $\hbar$ dependent {\em cobracket}
\[
\delta_{QFT}^1 : \fg[[z]] \to \fg[[z]] \tensor \fg[[z]] .
\]
We will show, that up to some scaling constant, that $\delta_{QFT}^1$ is that of Equation (\ref{LieBi}). 
By Drinfeld's result, this is enough to completely determine the Hopf algebra structure on $Y'(\fg)$. 

%\input{first section}

\bibliographystyle{alpha}
%\bibliographystyle{spmpsci}  
\bibliography{syllabus}

\end{document}